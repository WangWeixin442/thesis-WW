% !TEX root = ../thesis-WW.tex

% --------- FRONT MATTER PAGES ---------------------
% Title of the thesis
\title{Geometric Formulation of Uncertainties and Estimation \\ for Three-Dimensional Rotations}

% Author name
\author{Weixin Wang}

% Previous degrees
\bsdepartment{Mechanical Engineering}
\bsschool{Tsinghua University}
\bsgrad{May 2016}

\msdepartment{Mechanical Engineering}
\msschool{University of Wisconsin-Madison}
\msgrad{December 2018}
\showmsdegree % you can show or hide the MS degree line 
% \hidemsdegree

% PhD degree commands
% Committee
\showcommitteepage % hide this page if you're doing a MS thesis
%\hidecommitteepage 
\committee{ %
Dr. Taeyoung Lee, Professor of Mechanical and Aerospace Engineering,\\ 
George Washington University, Dissertation Director\\ % remember to add a space between committee members

Dr. Kausik Sarkar, Professor of Mechanical and Aerospace Engineering, \\
George Washington University, Committee Chair\\

Dr. Peng Wei, Assistant Professor of Mechanical and Aerospace Engineering, \\
George Washington University, Committee Member\\

Dr. Chung Hyuk Park, Associate Professor of Biomedical Engineering, \\
George Washington University, Committee Member\\

Dr. Kyle DeMars, Associate Professor of Aerospace Engineering, \\
Texas A\&M University, Committee Member\\
}

% Chair must be entered separately for formatting reasons.
\chair{Taeyoung Lee}
\chairtitle{Professor of Mechanical and Aerospace Engineering}

% Uncomment the following lines if there are two dissertation co-directors.
%\hascochair
%\cochair{Murray Snyder}
%\cochairtitle{Professor of Mechanical and Aerospace Engineering}

% Department
\department{Mechanical and Aerospace Engineering}

\phdgrad{July 21, 2022}
\defensedate{July 21, 2022}
% Year of completion for copyright page and perhaps other places
\year=2022

% Copyright page
%\copyrightholder{Someone else}

% Dedication
%\dedication{ %
%Include a fancy quote or dedication
%}

% Acknowledgments
\acknowledgments{
    Here you can acknowledge all of those people who have helped you to reach this point.
    It's rare that any work is done in a vacuum and your research is no exception.
    Feel free to be grateful for all those who've aided you along your way.
}

% -----------------------------------------------------------------
% Typically only one of Preface/Foreward/Prologue would be in your thesis.
% To choose one simply delete the others and they will automatically dissappear

% Preface
\preface{
    This is the preface. 
    It's another front matter page that offers additional detail into your work.
    Typically, only one (preface OR prologue OR foreword) is used. 
    You can remove the other sections by deleting them inside \texttt{tex/frontmatter.tex} or using the appropriate show or hide commands.
}

\prologue{
    This is the prologe. 
    It's another front matter page that offers additional detail into your work.
    Typically, only one (preface OR prologue OR foreword) is used. 
    You can remove the other sections by deleting them inside \texttt{tex/frontmatter.tex} or using the appropriate show or hide commands.
}

\foreword[2]{
    This is the forword. 
    It's another front matter page that offers additional detail into your work.
    Typically, only one (preface OR prologue OR foreword) is used. 
    You can remove the other sections by deleting them inside \texttt{tex/frontmatter.tex} or using the appropriate show or hide commands.
}
% ----------------------------------------------------------------------

% commands to show or hide front matter pages

\showcopyright
\showabstract
\showcommitteepage
\hidededication
\showacknowledgments
\hidepreface
\hideprologue
\hideforeword

% ------------ TABLE OF CONTENTS ----------------------
% Commands to hide or show lists of figures, tables, etc.
\showlistoffigures
\showlistoftables
\hidenomenclature

% --------- ACRONYMS and SYMBOLS ------------------------------
% TODO Deprecate the entire acronym package and switch to glossaries

% You can either use the acronymn or glossaries package (both work)
% Definition of any abbreviations used.
\abbreviations{
    \acro{CRTBP}{Circular Restricted Three Body Problem}
    \acro{NSA}{National Security Agency}
    \acro{SSME}{Space Shuttle Main Engine}
}
% call an abbreviation using \ac{abbrev}

% symbols and acronyms only show up when used in the text
\symbols{
    \acro{J}{Moment of Inertia}
}       

% if you want acronymn (simpler) then change these to show
\hidelistofabbreviations
\hidelistofsymbols

% if you want glossaries (more powerful) then leave above as hide
% GLOSSARIES package options - automatically turns off front pages from acronym package

% acronymns and symbols are basically the same, but there are two provided 
% locations where they can show up
\setabbreviationstyle[acronym]{long-short}
\setabbreviationstyle[abbreviation]{long-short}
\makeglossaries
% you can hide/show the glossaries page
\showglossarieslistofabbreviations
\showglossarieslistofsymbols
\showglossariesglossaryofterms

% acronyms defined in glossaries
% defining abbreviations like this allows for autocompletion
\newglossaryentry{IMU}{
    name={IMU},
    type=\glsxtrabbrvtype,
    description={Inertial Measurement Unit}
}

\newglossaryentry{GNSS}{
	name={GNSS},
	type=\glsxtrabbrvtype,
	description={Global Navigation Satellite System}
}

\newglossaryentry{MFG}{
	name={MFG},
	type=\glsxtrabbrvtype,
	description={Matrix Fisher--Gaussian Distribution}
}

\newglossaryentry{EKF}{
	name={EKF},
	type=\glsxtrabbrvtype,
	description={Extended Kalman Filter}
}

\newglossaryentry{MEKF}{
	name={MEKF},
	type=\glsxtrabbrvtype,
	description={Multiplicative Extended Kalman Filter}
}

\newglossaryentry{UKF}{
	name={UKF},
	type=\glsxtrabbrvtype,
	description={Unscented Kalman Filter}
}

\newglossaryentry{SLAM}{
	name={SLAM},
	type=\glsxtrabbrvtype,
	description={Simultaneous Localization and Mapping}
}

\newglossaryentry{IEKF}{
	name={IEKF},
	type=\glsxtrabbrvtype,
	description={Invariant Extended Kalman Filter}
}

\newglossaryentry{MSCKF}{
	name={MSCKF},
	type=\glsxtrabbrvtype,
	description={Multi-State Constraint Kalman Filter}
}

\newglossaryentry{BCH}{
	name={BCH},
	type=\glsxtrabbrvtype,
	description={Baker--Campbell--Hausdorff Formula}
}

\newglossaryentry{psvd}{
	name={pSVD},
	type=\glsxtrabbrvtype,
	description={proper Singular Value Decomposition}
}

\newglossaryentry{MLE}{
	name={MLE},
	type=\glsxtrabbrvtype,
	description={Maximum Likelihood Estimation}
}

\newglossaryentry{BG}{
	name={BG},
	type=\glsxtrabbrvtype,
	description={Bingham--Gaussian Distribution}
}

\newglossaryentry{MMSE}{
	name={MMSE},
	type=\glsxtrabbrvtype,
	description={Minimum Mean Square Estimate}
}


% glossary entries
\newglossaryentry{stiefel}{
    name=Stiefel manifold,
    description={An orthogonal $r$ frame in $\mathbb{R}^n$ with $r\leq n$.}
}

\newglossaryentry{normalizing}{
	name=normalizing constant,
	description={A constant for an un-normalized probability density function such that it integrates into one.}
}

\newglossaryentry{moment}{
	name=moment,
	description={The expectation of a random variable.}
}

\newglossaryentry{Haar}{
	name=Haar measure,
	description={The bi-invariant measure on $\SO{3}$ for integration of functions on $\SO{3}$. In is normalized such that $\SO{3}$ has unit volume.}
}

\newglossaryentry{inertial}{
	name=inertial frame,
	description={The reference frame that is assumed to be fixed in a navigation problem.}
}

\newglossaryentry{body-fixed}{
	name=body-fixed frame,
	description={The reference frame fixed to a rigid body.}
}

\newglossaryentry{recursive Bayesian filter}{
	name=recursive Bayesian filter,
	description={A stochastic estimator that estimates the state of a dynamical system by calculating the posterior distribution conditioned on the history of all measurements.}
}

\newglossaryentry{uncertainty propagation}{
	name=uncertainty propagation,
	description={Calculate how the probability density function of the state of a dynamical system evolves in time.}
}

\newglossaryentry{measurement update}{
	name=measurement update,
	description={Calculate the posterior density function from the prior and likelihood, using the Bayes' formula.}
}

\newglossaryentry{mle}{
	name=maximum likelihood estimation,
	description={Inference of the parameters of a probability distribution by maximizing the likelihood function of measurements.}
}

\newglossaryentry{kl}{
	name=KL-divergence,
	description={A metric indicating the difference between two probability distributions.}
}

\newglossaryentry{lm}{
	name=LM algorithm,
	description={An iterative gradient based algorithm to solve nonlinear minimum least square problems.}
}

% symbols
\newglossaryentry{SEN}{
	type=symbols,
	name={$\SE{n}$},
	sort=SEN,
	description={The $n$-dimensional special Euclidean group.}
}

\newglossaryentry{SON}{
    type=symbols,
    name={$\SO{n}$},
    sort=SON,
    description={The $n$-dimensional special orthogonal group.}
}

\newglossaryentry{SUN}{
	type=symbols,
	name={$\mathrm{SU}(n)$},
	sort=SUN,
	description={The $n$-dimensional special unitary group.}
}

\newglossaryentry{soN}{
	type=symbols,
	name={$\so{n}$},
	sort=soN,
	description={The Lie algebra of $\SO{n}$.}
}

\newglossaryentry{real}{
	type=symbols,
	name={$\mathbb{R}^n$},
	sort=R,
	description={The $n$-dimensional Euclidean space.}
}

\newglossaryentry{sph}{
	type=symbols,
	name={$\Sph^n$},
	sort=sph,
	description={Unit $n$-sphere.}
}

\newglossaryentry{int}{
	type=symbols,
	name={$\mathbb{Z}$},
	sort=z,
	description={The space of integers.}
}

\newglossaryentry{gamma}{
	type=symbols,
	name={$\Gamma(x)$},
	sort=gamma,
	description={Gamma function.}
}

\newglossaryentry{bessel}{
	type=symbols,
	name={$I_{\nu}(x)$},
	sort=Iv,
	description={Modified Bessel function of the first kind with order $\nu$.}
}

\newglossaryentry{hat}{
	type=symbols,
	name={$\hat{x}$},
	sort=hat,
	description={The hat map that identifies $\mathbb{R}^3$ to $\so{3}$.}
}

\newglossaryentry{vee}{
	type=symbols,
	name={$x^\vee$},
	sort=hat2,
	description={The vee map, inverse of the hat map.}
}

\newglossaryentry{exp}{
	type=symbols,
	name={$\exp(x)$},
	sort=exp,
	description={The exponential of a number $x\in\mathbb{R}$, or the exponential map of $x\in\so{3}$.}
}

\newglossaryentry{bracket}{
	type=symbols,
	name={$[A,B]$},
	sort=bracket,
	description={The Lie bracket for $A,B\in\so{3}$.}
}

\newglossaryentry{mf}{
	type=symbols,
	name={$\mathcal{M}(F)$},
	sort=matrixFisher,
	description={The matrix Fisher distribution with parameter $F$.}
}

\newglossaryentry{diag}{
	type=symbols,
	name={$\diag(x_1,\ldots,x_n)$},
	sort=diag,
	description={The (block) diagonal matrix with the diagonal terms given by $x_1,\ldots,x_n$.}
}

\newglossaryentry{vec}{
	type=symbols,
	name={$\mathrm{vec}(A)$},
	sort=vec,
	description={Concatenation of the columns of matrix $A$ into a column vector.}
}

\newglossaryentry{kron}{
	type=symbols,
	name={$\otimes$},
	sort=otime,
	description={Quaternion multiplication or Kronecker product depending on the context.}
}

\newglossaryentry{normal}{
	type=symbols,
	name={$\mathcal{N}(\mu,\Sigma)$},
	sort=normal,
	description={Normal distribution with mean $\mu$, and covariance matrix $\Sigma$.}
}

\newglossaryentry{von}{
	type=symbols,
	name={$\mathcal{VM}(\mu,\kappa)$},
	sort=von,
	description={The von Mises (Fisher) distribution with mean direction $\mu$, and concentration parameter $\kappa$.}
}

\newglossaryentry{MFGs}{
	type=symbols,
	name={$\mathcal{MG}(\mu,\allowbreak \Sigma,\allowbreak P,\allowbreak U,\allowbreak S,\allowbreak V)$},
	sort=MFG,
	description={The matrix Fisher--Gaussian distribution with parameters $(\mu,\Sigma,P,U,S,V)$.}
}

\newglossaryentry{bingham}{
	type=symbols,
	name={$\mathcal{B}(A)$},
	sort=bingham,
	description={The Bingham distribution with parameter $A$.}
}

\newglossaryentry{BGs}{
	type=symbols,
	name={$\mathcal{BG}(\mu,\Sigma,P,M,Z)$},
	sort=BG,
	description={The Bingham--Gaussian distribution with parameters $(\mu,\Sigma,P,M,Z)$.}
}

\newglossaryentry{phi}{
	type=symbols,
	name={$\varphi$},
	sort=phi,
	description={The homomorphism from $\Sph^3$ to $\SO{3}$.}
}

\newglossaryentry{trace}{
	type=symbols,
	name={$\tr{A}$},
	sort=trace,
	description={The trace of matrix $A$.}
}

\newglossaryentry{etr}{
	type=symbols,
	name={$\etr{A}$},
	sort=etr,
	description={An abbreviation of $\exp(\tr{A})$.}
}

\newglossaryentry{comb}{
	type=symbols,
	name={$\mathcal{C}(n,k)$},
	sort=comb,
	description={The set of all $k$-partitions of the set $\{1,2,\ldots,n\}$. For example $\mathcal{C}(3,2) = \Big\{ \big\{\{1\}, \allowbreak \{2,3\}\big\}, \allowbreak \big\{\{2\}, \allowbreak \{1,3\}\big\}, \allowbreak \big\{\{3\}, \allowbreak \{1,2\}\big\} \Big\}$}
}

\newglossaryentry{cS}{
	type=symbols,
	name={$c(S)$},
	sort=cS,
	description={The normalizing constant of the matrix Fisher distribution with parameter $S$.}
}

\newglossaryentry{ei}{
	type=symbols,
	name={$e_i$},
	sort=ei,
	description={The $i$-th column of $I_{n\times n}$, where $n$ depends on the context.}
}

\newglossaryentry{I}{
	type=symbols,
	name={$\mathcal{I}$},
	sort=I,
	description={The Fisher information.}
}

\newglossaryentry{expect}{
	type=symbols,
	name={$\expect{x}$},
	sort=expect,
	description={The expectation of a random element $x$.}
}

% Some abstract text
\abstract{
Modeling the uncertainty of three dimensional attitude is critical in numerous estimation problems in aerospace engineering and robotics.
However, the widely used Gaussian distribution is unable to accurately model large attitude uncertainties, due to the special geometric structure of the space for 3D rotations, namely the three dimensional special orthogonal group $\SO{3}$.
In this dissertation, the matrix Fisher distribution developed in directional statistics is used to model the attitude uncertainty.
The matrix Fisher distribution is defined intrinsically on $\SO{3}$, so it can accurately model arbitrarily large attitude uncertainties in a global fashion while fully respecting the underlying geometric structure of $\SO{3}$.

Although various properties of the matrix Fisher distribution have been studied in directional statistics, calculating its normalizing constant and moments still remains challenging.
In this dissertation, a recursive algorithm is proposed to calculate the central moment of matrix Fisher distribution up to an arbitrary order.
Also, an approximation for the matrix Fisher distribution is developed when it is highly concentrated in two degrees of freedom.
And based on this, approximate expressions for the normalizing constant and its derivatives are provided to reduce computation demand.

Next, a new probability density function on $\SO{3}\times \mathbb{R}^n$, referred to as the matrix Fisher--Gaussian distribution (MFG) is proposed to model the correlation between attitude and other random variables in the Euclidean space.
MFG greatly generalizes the application of matrix Fisher distribution, as in a lot of practical estimation problems, the attitude must be concurrently estimated with other quantities in the Euclidean space, and the correlation between them is crucial to transfer information from one to the other.
In addition, the Bingham--Gaussian distribution on $\Sph^3\times \mathbb{R}^n$ which is equivalent to MFG is constructed using the Lie group homomorphism from $\Sph^3$ to $\SO{3}$.
Various properties of these two distributions are studied, and a closed form approximate maximum likelihood estimation is presented for fast inference of their parameters required for real time implementations.

Finally, MFG is applied to three representative estimation problems, namely attitude estimation, IMU-GNSS integration, and visual-inertial odometry.
Specifically, the observability of attitude with angular velocity and single direction measurements is studied, where two new observable conditions are discovered.
Then, MFG is used to design a recursive Bayesian filter to estimate the attitude and gyroscope bias concurrently with attitude or direction measurements.
The uncertainty propagation step in this filter is further generalized into a full inertial navigation setting, where the estimated attitude is used to transform the accelerometer readings into the inertial frame which is integrated twice into position.
Based on this, a filter for IMU-GNSS navigation is developed with position measurements.
Furthermore, the MFG is also used to model the six dimensional pose that aligns two point sets.
This is combined with the uncertainty propagation scheme to design a visual-inertial odometry/navigation algorithm.
The proposed filters based on MFG are compared with the Gaussian distribution based multiplicative extended Kalman filter, where it is validated through simulation studies that the MFG filters have better accuracy in very challenging cases when the attitude has large uncertainty.
}
