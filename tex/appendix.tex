% !TEX root = ../thesis-WW.tex
\appendix
\doublespacing

\chapter{A Special Case in Theorem \ref{thm:MF-moment-dsdT}} \label{app:MF-moment-specialRecursion}

In this supplementary material, a special case which is embedded in the recursion of Theorem \eqref{thm:MF-moment-dsdT} is untwisted into a non-recursive formula except for an integer coefficient.

\begin{theorem} \label{thm:MF-moment-specialRecursion}
	If $s_1 \neq s_2 \neq |s_3|$ and $i \neq j$, then
	\begin{align} \label{eqn:dsisjdT}
		\left.\left( \frac{\partial^{2n}s'_i}{\partial T_{ij}^{2n}} + \frac{\partial^{2n}s'_j}{\partial T_{ij}^{2n}} \right)\right|_{T=0} &= \frac{a(n)}{(s_i+s_j)^{2n-1}}, \nonumber \\
		\left.\left( \frac{\partial^{2n}s'_i}{\partial T_{ij}^{2n}} - \frac{\partial^{2n}s'_j}{\partial T_{ij}^{2n}} \right)\right|_{T=0} &= \frac{a(n)}{(s_i-s_j)^{2n-1}},
	\end{align}
	where $a(n)$ is an integer-valued function of $n$.
\end{theorem}

Before proving this theorem, a lemma for differentiating $U'$ and $V'$ is needed.
\begin{lemma} \label{lemma:MF-moment-dUVdT-zero}
	Under the conditions of Theorem \ref{thm:MF-moment-dsdT}, if any of $k$ or $l$ is distinct from $i$ or $j$, then
	\begin{equation}
		\left. \frac{\partial^n U'_{kl}}{\partial T_{ij}^n} \right|_{T=0} = 0, \qquad
		\left. \frac{\partial^n V'_{kl}}{\partial T_{ij}^n} \right|_{T=0} = 0,
	\end{equation}
	for all $n>0$.
\end{lemma}
\begin{proof}
	Suppose $i=1,j=2$, and let $\begin{bmatrix} s_1 & T_{12} \\ 0 & s_2 \end{bmatrix} = \begin{bmatrix} U'_{11} & U'_{12} \\ U'_{21} & U'_{22} \end{bmatrix} \begin{bmatrix} s'_1 & 0 \\ 0 & s'_2 \end{bmatrix} \begin{bmatrix} V'_{11} & V'_{12} \\ V'_{21} & V'_{22} \end{bmatrix}^T$ be the singular value decomposition of the upper left 2-by-2 diagonal block of $S+T$.
	Then by direct calculation, $S+T$ has singular value decomposition
	\begin{equation*}
		\begin{bmatrix}
			s_1 & T_{12} & 0 \\
			0 & s_2 & 0 \\
			0 & 0 & s_3
		\end{bmatrix} = \begin{bmatrix}
			U'_{11} & U'_{12} & 0 \\
			U'_{21} & U'_{22} & 0 \\
			0 & 0 & 1
		\end{bmatrix} \begin{bmatrix}
			s'_1 & 0 & 0 \\
			0 & s'_2 & 0 \\
			0 & 0 & s_3
		\end{bmatrix} \begin{bmatrix}
			V'_{11} & V'_{12} & 0 \\
			V'_{21} & V'_{22} & 0 \\
			0 & 0 & 1
		\end{bmatrix}^T.
	\end{equation*}
	This means if the subscripts of $U'$ or $V'$ contain $3$ which is different from $\{1,2\}$, the corresponding element of $U'$ or $V'$ is either zero or one, whose derivatives with respect to $T_{12}$ of any order evaluated at $T=0$ is zero.
	Other cases can be shown similarly.
\end{proof}

Next, the proof for Theorem \ref{thm:MF-moment-specialRecursion} is provided.
\begin{proof}
	By direct calculation using \eqref{eqn:dsvd-dsdT}, is can be shown that
	\begin{align*}
		\frac{\partial (s'_i+s'_j)}{\partial T_{ij}} &= U'_{ii}V'_{ji} + U'_{ij}V'_{jj} \triangleq X_1, \\
		\frac{\partial (s'_i-s'_j)}{\partial T_{ij}} &= U'_{ii}V'_{ji} - U'_{ij}V'_{jj} \triangleq X_2,
	\end{align*}
	for $X_1,X_2\in\mathbb{R}$.
	Next, the second order differentiation is calculated using \eqref{eqn:dsvd-dUVdT} as
	\begin{align*}
		\frac{\partial X_1}{\partial T_{ij}}
		&= (U'_{ii}V'_{jj}-U'_{ij}V'_{ji})(\Omega_{U_{ij}}^{ij}+\Omega_{V_{ij}}^{ij}) + U'_{ik}V'_{ji}\Omega_{U_{ki}}^{ij} - U'_{ii}V'_{jk}\Omega_{V_{ki}}^{ij} + U'_{ik}V'_{jj}\Omega_{U_{kj}}^{ij} - U'_{ij}V'_{jk}\Omega_{V_{kj}}^{ij}, \\
		\frac{\partial X_2}{\partial T_{ij}} &= (U'_{ii}V'_{jj}+U'_{ij}V'_{ji})(\Omega_{V_{ij}}^{ij}-\Omega_{U_{ij}}^{ij}) + U'_{ik}V'_{ji}\Omega_{U_{ki}}^{ij} - U'_{ii}V'_{jk}\Omega_{V_{ki}}^{ij} - U'_{ik}V'_{jj}\Omega_{U_{kj}}^{ij} + U'_{ij}V'_{jk}\Omega_{V_{kj}}^{ij},
	\end{align*}
	where $k\in\{1,2,3\}$ and $i\neq j\neq k$.
	The crucial observation is that if we further differentiate the above two equations with respect to $T_{ij}$ and evaluate at $T=0$, then the last four terms in each subequation yield terms involving either $U'_{ik}$, $V'_{jk}$, or their derivatives with respect to $T_{ij}$.
	However, Lemma \ref{lemma:MF-moment-dUVdT-zero} indicates that all these terms vanish after evaluated at $T=0$.
	This means when calculating \eqref{eqn:dsisjdT}, the last four terms in each of the above subequations can be simply omitted.
	Therefore, by \eqref{eqn:dsvd-OmegaUV} is can be shown that
	\begin{align*}
		\frac{\partial X_1}{\partial T_{ij}} &= (U'_{ii}V'_{jj}-U'_{ij}V'_{ji})(\Omega_{U_{ij}}^{ij}+\Omega_{V_{ij}}^{ij}) = \frac{(U'_{ii}V'_{jj}-U'_{ij}V'_{ji})^2}{s'_i+s'_j} \triangleq \frac{Y_1^2}{s'_i+s'_j}, \\
		\frac{\partial X_2}{\partial T_{ij}} &= (U'_{ii}V'_{jj}+U'_{ij}V'_{ji})(-\Omega_{U_{ij}}^{ij}+\Omega_{V_{ij}}^{ij}) = \frac{(U'_{ii}V'_{jj}+U'_{ij}V'_{ji})^2}{s'_i-s'_j} \triangleq \frac{Y_2^2}{s'_i-s'_j}.
	\end{align*}
	If the above two equations are differentiated with respect to $T_{ij}$ again, we obtain the derivatives of $s_i \pm s_j$ which have been calculated, and the derivatives of $Y_1$, $Y_2$ which are calculated as follows.
	\begin{align*}
		\frac{\partial Y_1}{\partial T_{ij}} &= -\frac{(V'_{ii}V'_{ji}+U'_{ij}V'_{jj})(U'_{ii}V'_{jj}-U'_{ij}V'_{ji})}{s'_i+s'_j} = -\frac{X_1Y_1}{s'_i+s'_j}, \\
		\frac{\partial Y_2}{\partial T_{ij}} &= -\frac{(U'_{ii}V'_{ji}-U'_{ij}V'_{ji})(U'_{ii}V'_{jj}+U'_{ij}V'_{ji})}{s'_i-s'_j} = -\frac{X_2Y_2}{s'_i-s'_j},
	\end{align*}
	where any term having the subscript $k$ has been omitted. 
	
	Now, a very special branch of smaller recursion is distilled from the large recursion stated in Theorem \ref{thm:MF-moment-dsdT}, which reads
	\begin{align} \label{eqn:smallRecursion}
		\frac{\partial (s'_i+s'_j)}{\partial T_{ij}} &= X_1, \qquad \frac{\partial X_1}{\partial T_{ij}} = \frac{Y_1^2}{s'_i+s'_j}, \qquad \frac{\partial Y_1}{\partial T_{ij}} = -\frac{X_1Y_1}{s'_i+s'_j}, \nonumber \\
		\frac{\partial (s'_i-s'_j)}{\partial T_{ij}} &= X_2, \qquad \frac{\partial X_2}{\partial T_{ij}} = \frac{Y_2^2}{s'_i-s'_j}, \qquad \frac{\partial Y_2}{\partial T_{ij}} = -\frac{X_2Y_2}{s'_i-s'_j}.
	\end{align}
	Since the recursive structure for $s'_i+s'_j$ and $s'_i-s'_j$ are the same, they can be written in the same form as
	\begin{equation*}
		\frac{\partial s'_{ij}}{\partial T_{ij}} = X; \qquad \frac{\partial X}{\partial T_{ij}} = \frac{Y^2}{s'_{ij}}; \qquad \frac{\partial Y}{\partial T_{ij}} = -\frac{XY}{s'_{ij}},
	\end{equation*}
	where $s'_{ij}$ denotes either $s'_i+s'_j$ or $s'_i-s'_j$.
	Finally, to prove the theorem, it is claimed that
	\begin{equation*}
		\frac{\partial^{2n} s'_{ij}}{\partial T_{ij}^{2n}} = \frac{a_0Y^{2n} + a_1Y^{2n-2}X^2 + \cdots + a_{n-1}Y^2X^{2n-2}}{{s'_{ij}}^{2n-1}},
	\end{equation*}
	where $a_m$ is an integer coefficient for $m=0,\ldots,n-1$.
	This claim is proved by induction.
	When $n=1$, it is clearly seen from \eqref{eqn:smallRecursion}.
	Suppose for case $n$ it is true, then differentiate $T_{ij}$ one more time, and after some rearrangement, it is shown that
	\begin{equation*}
		\frac{\partial^{2n+1} s'_{ij}}{\partial T_{ij}^{2n+1}} = \frac{a'_0Y^{2n}X + a'_1Y^{2n-2}X^3 + \cdots + a'_{n-1}Y^2X^{2n-1}}{\partial {s'_{ij}}^{2n}},
	\end{equation*}
	where $a'_m = (-4n+2m+1)a_m + 2(m+1)a_{m+1}$ for $m=0,\ldots,n-2$, and $a'_{n-1} = (-2n-1)a_{n-1}$.
	Then, differentiate $T_{ij}$ once again, it is shown that
	\begin{align*}
		\frac{\partial^{2n+2} s'_{ij}}{\partial T_{ij}^{2n+2}} = \frac{a''_0Y^{2n+2} + a''_1Y^{2n}X^2 + \cdots + a''_{n}Y^2X^{2n}}{{s'_{ij}}^{2n+1}},
	\end{align*}
	where $a''_0 = a'_0$, $a''_m = (-4n+2m-2)a'_{m-1} + (2m+1)a'_m$ for $m=1,\ldots,n-1$, and $a''_n = (-2n-2)a'_{n-1}$.
	This finishes the proof for the claim.
	Now, note that when evaluated at $T=0$, we have $Y=1$, $X=0$.
	Thus $\left. \partial^{2n}s'_{ij} / \partial T_{ij}^{2n} \right|_{T=0} = a_0 / s_{ij}^{2n-1}$, which finishes the proof for Theorem \ref{thm:MF-moment-specialRecursion} by noting that $a_0$ is an integer function of $n$.
\end{proof}

It should be noted that the idea in the proof of Theorem \ref{thm:MF-moment-specialRecursion} cannot be generalized to the large recursion stated in Theorem \ref{thm:MF-moment-dsdT}, due to the omitted terms in the derivatives of $X$ and $Y$.
Even in this simpler case, giving a non-recursive expression for $a(n)$ is complicated, as seen from the recursive formula given for $a_m(n)$ in the proof.

\chapter{The Second and Third Order Moment of Matrix Fisher Distribution} \label{app:MF-moment-second-third}

In this Appendix, non-recursive expressions of the second and third order moments are provided based on the development in Chapter \ref{section:MF-moments}.

\section{The Second Order Moments}

Based on Lemma \ref{lemma:MF-moment-EQii}, the second order moment of the form $\expect{Q_{ii}Q_{jj}}$ can be calculated as
\begin{align} \label{eqn:MF-moment-EQiijj}
	\expect{Q_{ii}Q_{jj}} = \frac{1}{c(S)} \frac{\partial^2 c(S)}{\partial s_i \partial s_j},
\end{align}
for any $i,j\in\{1,2,3\}$.
Next, using the recursion developed in Theorem \ref{thm:MF-moment-dsdT}, the second order moment of the forms $\expect{Q_{ij}Q_{ij}}$ and $\expect{Q_{ij}Q_{ji}}$ can be calculated as
\begin{subequations} \label{eqn:MF-moment-EQijij-EQijji}
	\begin{align}
		\expect{Q_{ij}Q_{ij}} &= \frac{1}{c(S)}\left(-\frac{\partial c(S)}{\partial s_i}\frac{s_i}{s_j^2-s_i^2}+\frac{\partial c(S)}{\partial s_j}\frac{s_j}{s_j^2-s_i^2}\right), \\
		\expect{Q_{ij}Q_{ji}} &= \frac{1}{c(S)}\left(-\frac{\partial c(S)}{\partial s_i}\frac{s_j}{s_j^2-s_i^2}+\frac{\partial c(S)}{\partial s_j}\frac{s_i}{s_j^2-s_i^2}\right),
	\end{align}
\end{subequations}
for any $i\neq j \in \{1,2,3\}$, and $|s_i|\neq |s_j|$.

If $s_i = s_j \neq 0$, \eqref{eqn:MF-moment-EQijij-EQijji} can be evaluated by taking the limit $s_j\to s_i$, and their explicit expressions are
\begin{subequations} \label{eqn:MF-moment-EQijij-EQijji-degenerate1}
	\begin{align}
		\expect{Q_{ij}Q_{ij}} &= \frac{1}{c(S)}\frac{1}{2s_i} \left( \frac{\partial c(S)}{\partial s_i} + s_i\left( \frac{\partial^2 c(S)}{\partial s_i^2} - \frac{\partial^2 c(S)}{\partial s_i \partial s_j} \right) \right), \\
		\expect{Q_{ij}Q_{ji}} &= \frac{1}{c(S)}\frac{1}{2s_i}\left( -\frac{\partial c(S)}{s_i} + s_i\left( \frac{\partial^2 c(S)}{\partial s_i^2} - \frac{\partial^2 c(S)}{\partial s_i \partial s_j} \right) \right).
	\end{align}
\end{subequations}
Similarly, if $s_i = -s_j \neq 0$, their expressions become
\begin{subequations}
	\begin{align}
		\expect{Q_{ij}Q_{ij}} &= \frac{1}{c(S)} \frac{1}{2s_i} \left( \frac{\partial c(S)}{\partial s_i}  + s_i\left( \frac{\partial^2 c(S)}{\partial s_i^2} + \frac{\partial^2 c(S)}{\partial s_i \partial s_j} \right) \right), \\
		\expect{Q_{ij}Q_{ji}} &= \frac{1}{c(S)} \frac{1}{2s_i} \left( \frac{\partial c(S)}{\partial s_i} - s_i\left( \frac{\partial^2 c(S)}{\partial s_i^2} + \frac{\partial^2 c(S)}{\partial s_i \partial s_j} \right) \right).
	\end{align}
\end{subequations}
If $s_i = s_j = 0$, the above equations can also be evaluated by taking the limit $s_i\to 0$, as
\begin{subequations} \label{eqn:MF-moment-EQijij-EQijji-degenerate3}
	\begin{align}
		\expect{Q_{ij}Q_{ij}} &= \frac{1}{c(S)} \frac{\partial^2 c(S)}{\partial s_i^2}, \\
		\expect{Q_{ij}Q_{ji}} &= -\frac{1}{c(S)} \frac{\partial^2 c(S)}{\partial s_i \partial s_j}.
	\end{align}
\end{subequations}

Given these results and Theorem \ref{thm:MF-moment-dcds}, the second order derivatives of the normalizing constant can be easily evaluated by solving a linear system using the first order derivatives.
This result is summarized in the following theorem.
\begin{theorem} \label{thm:MF-moment-dcds-second}
	Let $\partial^2 c(S) \in \mathbb{R}^{3\times 3}$ be a matrix whose elements are the second order derivatives $\partial^2 c(S)/\partial s_i \partial s_j$, $i,j\in\{1,2,3\}$.
	Then $\partial^2 c(S)$ satisfies a linear system $A \cdot \mathrm{vec}(\partial^2 c(S)) = b$, where $A\in\mathbb{R}^{9\times 9}$ is constant, and $b\in\mathbb{R}^9$ only involves $S$, $c(S)$, and the first order derivatives of $c(S)$.
\end{theorem}
\begin{proof}
	According to \eqref{eqn:MF-moment-EQiijj}, it can be shown that
	\begin{align*}
		\frac{\partial^2 c(S)}{\partial s_i^2} = c(S)\expect{Q_{ii}^2} = c(S)\left(1-\expect{Q_{ij}^2}-\expect{Q_{ik}^2}\right),
	\end{align*}
	where $i\neq j\neq k$.
	If $s_j\neq s_k$, then by \eqref{eqn:MF-moment-EQijij-EQijji} and the above equation, $\partial^2 c(S)/\partial s_i^2$ can be written as an expression involved with $S$, $c(S)$, and the first order derivatives of $c(S)$.
	If $s_j=s_k\neq 0$, or $s_j=-s_k\neq 0$, or $s_j=s_k=0$, substitute \eqref{eqn:MF-moment-EQijij-EQijji-degenerate1} to \eqref{eqn:MF-moment-EQijij-EQijji-degenerate3} into the above equation.
	Then the coefficient of $\partial^2 c(S)/\partial s_i^2$, $\partial^2 c(S)/\partial s_i \partial s_j$, and $\partial^2 c(S)/\partial s_i \partial s_k$ on the right hand side are constants.
	After moving them to the left hand side, the right hand side only has $S$, $c(S)$, and the first order derivatives of $c(S)$.
	These provides three equations in total.
	
	Next, by \eqref{eqn:MF-moment-EQiijj} and \eqref{eqn:SO3-Rkk}, it can be shown that
	\begin{align*}
		\frac{\partial^2 c(S)}{\partial s_i \partial s_j} = c(S)\expect{Q_{ii}Q_{kk}} = c(S)\left(\expect{Q_{kk}}+\expect{Q_{ij}Q_{ji}}\right),
	\end{align*}
	for $i\neq j\neq k$.
	Substitute \eqref{eqn:MF-S2D} and one from \eqref{eqn:MF-moment-EQijij-EQijji} to \eqref{eqn:MF-moment-EQijij-EQijji-degenerate3} into the above equation, and move $\partial^2 c(S)/\partial s_i^2$, $\partial^2 c(S)/\partial s_i \partial s_j$ from the right had side to the left hand side, the remaining six equations are obtained.
\end{proof}

\section{The Third Order Moments}

Based on Lemma \ref{lemma:MF-moment-EQii}, the third order moments of the form
$\expect{Q_{ii}Q_{jj}Q_{kk}}$ are
\begin{align}
	\expect{Q_{ii}Q_{jj}Q_{kk}} = \frac{1}{c(S)} \frac{\partial^3 c(S)}{\partial s_i \partial s_j \partial s_k},
\end{align}
for $i,j,k\in\{1,2,3\}$.
Next, moments of the form $\expect{Q_{ii}Q_{jk}Q_{kj}}$ are
\begin{subequations}
	\begin{align}
		\expect{Q_{ii}Q_{jk}Q_{jk}} &= \frac{1}{c(S)} \left( \frac{\partial^2 c(S)}{\partial s_i \partial s_k} \frac{s_k}{s_k^2-s_j^2} - \frac{\partial^2 c(S)}{\partial s_i \partial s_j} \frac{s_j}{s_k^2-s_j^2} \right), \\
		\expect{Q_{ii}Q_{jk}Q_{kj}} &= \frac{1}{c(S)} \left( \frac{\partial^2 c(S)}{\partial s_i \partial s_k} \frac{s_j}{s_k^2-s_j^2} - \frac{\partial^2 c(S)}{\partial s_i \partial s_j} \frac{s_k}{s_k^2-s_j^2} \right),
	\end{align}
\end{subequations}
for $i\neq j\neq k$ and $|s_j|\neq |s_k|$.
Moments of the form $\expect{Q_{jj}Q_{jk}Q_{jk}}$ are
\begin{subequations}
	\begin{align}
		\expect{Q_{jj}Q_{jk}Q_{jk}} &= \frac{1}{c(S)} \left( \frac{\partial^2 c(S)}{\partial s_j \partial s_k} \frac{s_k}{s_k^2-s_j^2} - \frac{\partial^2 c(S)}{\partial s_j^2} \frac{s_j}{s_k^2-s_j^2} \right. \nonumber \\
		&\qquad\qquad \left. - \frac{\partial c(S)}{\partial s_j} \frac{s_j^2+s_k^2}{(s_j^2-s_k^2)^2} + \frac{\partial c(S)}{\partial s_k} \frac{2s_js_k}{(s_j^2-s_k^2)^2} \right), \\
		\expect{Q_{jj}Q_{jk}Q_{kj}} &= \frac{1}{c(S)} \left( \frac{\partial^2 c(S)}{\partial s_j \partial s_k} \frac{s_j}{s_k^2-s_j^2} - \frac{\partial^2 c(S)}{\partial s_j^2} \frac{s_k}{s_k^2-s_j^2} \right. \nonumber \\
		&\qquad\qquad \left. - \frac{\partial c(S)}{\partial s_j} \frac{2s_js_k}{(s_j^2-s_k^2)^2} + \frac{\partial c(S)}{\partial s_k} \frac{s_j^2+s_k^2}{(s_j^2-s_k^2)^2} \right),
	\end{align}
\end{subequations}
for $j\neq k$ and $|s_j|\neq |s_k|$.
Finally, moments of the form $\expect{Q_{ij}Q_{jk}Q_{ki}}$ are
\begin{subequations}
	\begin{align}
		\expect{Q_{ij}Q_{jk}Q_{ki}} &= \frac{1}{c(S)}\left( \frac{\partial c(S)}{\partial s_i} \frac{s_js_k}{(s_i^2-s_j^2)(s_i^2-s_k^2)} \right. \nonumber \\
		&\qquad\qquad \left. + \frac{\partial c(S)}{\partial s_j} \frac{s_is_k}{(s_j^2-s_i^2)(s_j^2-s_k^2)} + \frac{\partial c(S)}{\partial s_k} \frac{s_is_j}{(s_k^2-s_i^2)(s_k^2-s_j^2)} \right), \\
		\expect{Q_{ij}Q_{jk}Q_{ik}} &= \frac{1}{c(S)}\left( \frac{\partial c(S)}{\partial s_i} \frac{s_is_j}{(s_j^2-s_i^2)(s_k^2-s_i^2)} \right. \nonumber \\
		&\qquad\qquad \left. + \frac{\partial c(S)}{\partial s_j} \frac{s_j^2}{(s_i^2-s_j^2)(s_k^2-s_j^2)} + \frac{\partial c(S)}{\partial s_k} \frac{s_js_k}{(s_j^2-s_k^2)(s_i^2-s_k^2)} \right),
	\end{align}
\end{subequations}
for $i\neq j\neq k$ and $|s_i|\neq |s_j| \neq |s_k|$.

\chapter{Non-uniqueness Parameters of MFG} \label{app:MFG-unique}

In this appendix, it is shown that how the MFG can be parameterized differently if $S$ has repeated values.
First, Definition \ref{def:psvd} is augmented by the following uniqueness condition:
\textit{the first nonzero element of each column of $U'$ is positive} \cite{khatri1977mises}.
This condition ensures that the columns of $U$ and $V$ cannot undergo simultaneous sign changes.
Then the equivalence of MFG parameterizations is given in the following theorem.
\begin{theorem} \label{thm:MFG-equivalent}
	Suppose $F=USV$, $\tilde{F}=\tilde{U}\tilde{S}\tilde{V}^T$ are the proper SVD of $F$ and $\tilde{F}$ with the augmented uniqueness condition.
	Then  $\mathcal{MG}(\mu,\allowbreak \Sigma,\allowbreak P,\allowbreak U,\allowbreak S,\allowbreak V)$ and $\mathcal{MG}(\tilde{\mu},\allowbreak \tilde{\Sigma},\allowbreak \tilde{P},\allowbreak \tilde{U},\allowbreak \tilde{S},\allowbreak \tilde{V})$ are equivalent if and only if $\mu=\tilde{\mu}$, $S=\tilde{S}$, and one of the following conditions is satisfied:
	\begin{enumerate}
		\item if $s_1=s_2=s_3=0$, then $\Sigma = \tilde{\Sigma}$. \label{case:s1=s2=s3=0}
		\item if $s_1 \neq s_2=s_3=0$, then \label{case:s1 neq s2=s3=0}
		\begin{enumerate}
			\item [2I.] $\exists \theta_1, \theta_2 \in \mathbb{R}$ such that $\tilde{U}=UT_1$, $\tilde{V}=VT_2$ where $T_1 = \exp(\theta_1\hat{e}_1)$ and $T_2 = \exp(\theta_2\hat{e}_1)$, $[\tilde{P}_{:,2},\tilde{P}_{:,3}] = [P_{:,2},P_{:,3}]\begin{bmatrix} \cos\theta_1 & -\sin\theta_1 \\ \sin\theta_1 & \cos\theta_1\end{bmatrix}$ where $P_{:,i}$ is the $i$-th column of $P$, and $\Sigma = \tilde{\Sigma}$.
			\item [2B.] $\exists \theta_1, \theta_2 \in \mathbb{R}$ such that $\tilde{U}=UT_1$, $\tilde{V}=VT_2$ where $T_1 = \exp(\theta_1\hat{e}_1)$ and $T_2 = \exp(\theta_2\hat{e}_1)$, $[\tilde{P}_{:,2},\tilde{P}_{:,3}] = [P_{:,2},P_{:,3}]\begin{bmatrix} \cos\theta_2 & -\sin\theta_2 \\ \sin\theta_2 & \cos\theta_2\end{bmatrix}$ where $P_{:,i}$ is the $i$-th column of $P$, and $\Sigma = \tilde{\Sigma}$.
		\end{enumerate} 
		\item if $s_1=s_2=s_3 \neq 0$, then $\exists T\in\SO{3}$ such that $\tilde{U}=UT$, $\tilde{V}=VT$, $\tilde{P}=PT$ and $\Sigma=\tilde{\Sigma}$.\label{case:s1=s2=s3 neq 0}
		\item if $s_1 \neq s_2=s_3 \neq 0$, then $\exists \theta\in\mathbb{R}$ such that $\tilde{U}=UT$, $\tilde{V}=VT$, $\tilde{P}=PT$, where $T=\exp(\theta\hat{e}_1)$, and $\Sigma=\tilde{\Sigma}$.\label{case:s1 neq s2=s3 neq 0}
		\item if $s_1=s_2 \neq |s_3|$, then $\exists \theta\in\mathbb{R}$ such that $\tilde{U}=UT$, $\tilde{V}=VT$, $\tilde{P}=PT$, where $T=\exp(\theta\hat{e}_3)$, and $\Sigma=\tilde{\Sigma}$.\label{case:s1=s2 neq |s3|}
		\item if $s_1 \neq s_2 \neq |s_3|$, then $U=\tilde{U}$, $V=\tilde{V}$, $P=\tilde{P}$, and $\Sigma=\tilde{\Sigma}$. \label{case:s1 neq s2 neq |s3|}
		\item if $s_1=s_2=-s_3 \neq 0$, let $L = \diag(1,1,-1)$, then \label{case:s1=s2=-s3 neq 0}
		\begin{enumerate}
			\item [7I.] $\exists T\in\SO{3}$ such that $\tilde{U}=UT$, $\tilde{V}=VLTL$, $\tilde{P}=PT$, and \linebreak $\tilde{\Sigma} = \Sigma + P\big[T(\tr{S}I_{3\times 3}-S)T^T - (\tr{S}I_{3\times 3}-S)\big]P^T$.
			\item [7B.] $\exists T\in\SO{3}$ such that  $\tilde{U}=ULTL$, $\tilde{V}=VT$, $\tilde{P}=PT$, and \linebreak $\tilde{\Sigma} = \Sigma + P\big[ T(\tr{S}I_{3\times 3}-S)T^T - (\tr{S}I_{3\times 3}-S) \big]P^T$.
		\end{enumerate}
		\item if $s_1 \neq s_2=-s_3 \neq 0$, then \label{case:s1 neq s2=-s3 neq 0}
		\begin{enumerate}
			\item [8I.] $\exists \theta\in\mathbb{R}$ such that $\tilde{U}=UT$, $\tilde{V}=VT^T$, $\tilde{P}=PT$, and $\tilde{\Sigma} = \Sigma + P\big[ T(\tr{S}I_{3\times 3}-S)T^T - (\tr{S}I_{3\times 3}-S) \big]P^T$, where $T=\exp(\theta\hat{e}_1)$.
			\item [8B.] $\exists \theta\in\mathbb{R}$ such that $\tilde{U}=UT^T$, $\tilde{V}=VT$, $\tilde{P}=PT$, and $\tilde{\Sigma} = \Sigma + P\big[ T(\tr{S}I_{3\times 3}-S)T^T - (\tr{S}I_{3\times 3}-S) \big]P^T$, where $T=\exp(\theta\hat{e}_1)$.
		\end{enumerate}
	\end{enumerate}
	For cases \ref{case:s1 neq s2=s3=0}), \ref{case:s1=s2=s3 neq 0}) and \ref{case:s1 neq s2=-s3 neq 0}), ``$I$'' means the condition holds for MFGI, and ``$B$'' means the condition holds for MFGB.
\end{theorem}
\begin{proof}
	By Lemma \ref{lemma:MFG-equivalent-intermediate}, the two MFGs are equivalent if and only if $F=\tilde{F}$, $\mu_c=\tilde{\mu}_c$ for all $R\in\SO{3}$, and $\Sigma=\tilde{\Sigma}_c$.
	
	The conditions provided in the theorem begin with $S=\tilde {S}$. 
	We first consider additional conditions to make them equivalent to $F=\tilde{F}$.
	Since a matrix cannot have two sets of different singular values, $F=\tilde{F}$ implies $S=\tilde{S}$.
	For case \ref{case:s1=s2=s3=0}), $S=\tilde{S}$ trivially implies $F=\tilde {F}$ as they are all zeros.
	For case \ref{case:s1 neq s2 neq |s3|}), $F=\tilde{F}$ if and only if their proper SVDs are the same, due to the augmented uniqueness condition for SVD when the singular values are different.
	In short, $F=\tilde{F}$ if and only if $S=\tilde{S}$ for cases 1) and 6).
	However, when $F$ has repeated singular values, $U$ and $V$ are only unique up to a rotation.
	We consider three possibilities regarding the multiplicity of $S$.
	
	(i) If $s_1 \neq s_2=s_3=0$, corresponding to case \ref{case:s1 neq s2=s3=0}) in the theorem, then $F=\tilde{F}$ if and only if $U\diag(s_1,0,0)V^T = \tilde{U}\diag(s_1,0,0)\tilde{V}^T$.
	This means the first column of $U$ and $V$ is unique, while other columns can be arbitrarily chosen as long as $U,V\in\SO{3}$, which is the same as the conditions on $U,V$ in case \ref{case:s1 neq s2=s3=0}).
	
	(ii) If $s_1=s_2$ and(or) $s_2=s_3 \neq 0$, which corresponds to case \ref{case:s1=s2=s3 neq 0}) to \ref{case:s1=s2 neq |s3|}) in the theorem, then the left and right singular vectors (columns of $U$ and $V$) w.r.t the repeated singular value form subspaces of $\mathbb{R}^3$.
	And $F=\tilde{F}$ if and only if the left and right singular vectors w.r.t the repeated singular value are rotated by the same rotation matrix, while the singular vectors w.r.t the non-repeated singular value are the same.
	The above condition on $U,V$ is the same as those given in case \ref{case:s1=s2=s3 neq 0}) to \ref{case:s1=s2 neq |s3|}).
	
	(iii) If $s_2=-s_3 \neq 0$, corresponding to cases \ref{case:s1=s2=-s3 neq 0}) and \ref{case:s1 neq s2=-s3 neq 0}) in the theorem.
	Let $S' = SL \triangleq \diag(s'_1,s'_2,s'_3)$, then $USV^T = US'LV^T = ULS'V^T$.
	Since $s'_2=s'_3$, for all $\theta\in\mathbb{R}$ and $T=\exp(\theta\hat{e}_1)$, it can be proved for MFGI that
	\begin{align*}
		US'LV^T = UTS'T^TLV^T = UTS'LLT^TLV^T = UTS(VLTL)^T = UTS(VT^T)^T,
	\end{align*}
	and for MFGB that
	\begin{align*}
		ULS'V^T = ULTS'T^TV^T = ULTLLS'T^TV^T = (ULTL)S(VT)^T = UT^TS(VT)^T,
	\end{align*}
	This shows the sufficiency and necessity for the conditions on $U,V$ in case \ref{case:s1 neq s2=-s3 neq 0}).
	For case \ref{case:s1=s2=-s3 neq 0}), $T$ can be arbitrarily chosen from $\SO{3}$, but in general $LTL \neq T^T$.
	
	Next, we consider the equivalent conditions for $\mu_c=\tilde{\mu}_c$ and $\Sigma_c=\tilde{\Sigma}_c$.
	For case \ref{case:s1=s2=s3=0}), the matrix Fisher density parts reduce to one, $\mu_c = \mu$, and $\Sigma_c=\Sigma$, so the remaining Gaussian density parts are equivalent if and only if their means and covariance matrices are the same, as shown in Lemma \ref{lemma:MFG-equivalent-intermediate}.
	Next, the three possibilities regarding the multiplicity of $S$ are considered individually as follows.
	
	For (i), let $Q=U^TRV$, since $\tilde{U}=UT_1$, $\tilde{V}=VT_2$ and $S=\tilde{S}$, it can be proved for MFGI that
	\begin{align} \label{eqn:MFG-equivalent-Miuc-MFGI}
		\tilde{\mu}_c &= \tilde{\mu} + \tilde{P}(T_1^TU^TRVT_2S-ST_2^TV^TR^TUT_1)^\vee \nonumber \\
		&= \tilde{\mu} + \tilde{P}T_1^T(U^TRVT_2ST_1^T-T_1ST_2^TV^TR^TU)^\vee \nonumber \\
		&= \tilde{\mu} + \tilde{P}T_1^T(QS-SQ^T)^\vee \nonumber \\
		&= \tilde{\mu} + \tilde{P}T_1^T[0, -s_1Q_{31}, s_1Q_{21}]^T.
	\end{align}
	And similarly, for MFGB, the above equation becomes
	\begin{align} \label{eqn:MFG-equivalent-Miuc-MFGB}
		\tilde{\mu}_c &= \tilde{\mu} + \tilde{P}(ST_1^TU^TRVT_2-T_2^TV^TR^TUT_1S)^\vee \nonumber \\
		&= \tilde{\mu} + \tilde{P}T_2^T(T_2ST_1^TU^TRV-V^TR^TUT_1ST_2^T)^\vee \nonumber \\
		&= \tilde{\mu} + \tilde{P}T_2^T(SQ-Q^TS)^\vee \nonumber \\
		&= \tilde{\mu} + \tilde{P}T_2^T[0, s_1Q_{13}, -s_1Q_{12}]^T.
	\end{align}
	Note that $Q$ and $R$ have a one-to-one correspondence, so $\mu_c = \tilde{\mu}_c$ for all $R\in\SO{3}$ is equivalent to $\mu_c = \tilde{\mu}_c$ for all $Q\in\SO{3}$.
	It is clear the conditions on $\mu$ and $P$ in case \ref{case:s1 neq s2=s3=0}) are sufficient for this.
	On the other hand, if $\mu_c = \tilde{\mu}_c$ for all $Q\in\SO{3}$, substituting $Q=I_{3 \times 3}$ into \eqref{eqn:MFG-equivalent-Miuc-MFGI} and \eqref{eqn:MFG-equivalent-Miuc-MFGB} proves $\mu=\tilde{\mu}$; substituting $Q = \expb{\tfrac{\pi}{2}e_2}$ and $Q = \expb{\tfrac{\pi}{2}e_3}$ into \eqref{eqn:MFG-equivalent-Miuc-MFGI} and \eqref{eqn:MFG-equivalent-Miuc-MFGB} proves $[\tilde{P}_{:,2},\tilde{P}_{:,3}] = [P_{:,2},P_{:,3}]\begin{bmatrix} \cos\theta_1 & -\sin\theta_1 \\ \sin\theta_1 & \cos\theta_1\end{bmatrix}$.
	Finally, denote $T'$ as either $\begin{bmatrix} \cos\theta_1 & -\sin\theta_1 \\ \sin\theta_1 & \cos\theta_1\end{bmatrix}$ or $\begin{bmatrix} \cos\theta_2 & -\sin\theta_2 \\ \sin\theta_2 & \cos\theta_2\end{bmatrix}$ then
	\begin{align}
		\tilde{P}(\tr{S}I_{3 \times 3}-S)(\tilde{P})^T = s_1 \begin{bmatrix} P_{:,2} & P_{:,3} \end{bmatrix}  T'(T')^T \begin{bmatrix} P_{:,2}^T \\  P_{:,3}^T \end{bmatrix} = P(\tr{S}I_{3 \times 3}-S)P^T.
	\end{align}
	so $\Sigma_c = \tilde{\Sigma}_c$ if and only if $\Sigma = \tilde{\Sigma}$.
	
	For (ii), a similar calculation as in \eqref{eqn:MFG-equivalent-Miuc-MFGI} and \eqref{eqn:MFG-equivalent-Miuc-MFGB} shows $\tilde{\mu}_c = \tilde{\mu} + \tilde{P}T^T \nu_R$.
	The sufficiency of the conditions on $\mu$ and $P$ in case \ref{case:s1=s2=s3 neq 0}) to \ref{case:s1=s2 neq |s3|}) follows immediately.
	For the necessary direction, substituting $Q = I_{3 \times 3}$ into the above equation proves $\mu = \tilde{\mu}$;
	substituting $Q = \expb{\tfrac{\pi}{2}\hat{e}_1}$, $Q = \expb{\tfrac{\pi}{2}\hat{e}_2}$ and $Q = \expb{\tfrac{\pi}{2}\hat{e}_3}$ proves $\tilde{P} = PT$ since $s_i+s_j>0$ for any $i \neq j$ in case \ref{case:s1=s2=s3 neq 0}) to \ref{case:s1=s2 neq |s3|}).
	Finally, note that the repeated diagonal entries of $S$ and $\tr{S}I_{3 \times 3}-S$ share the same indices, thus $T(\tr{S}I_{3 \times 3}-S)T^T = \tr{S}I_{3 \times 3}-S$.
	This proves $\Sigma_c = \tilde{\Sigma}_c$ if and only if $\Sigma = \tilde{\Sigma}$.
	The same argument in this paragraph also proves case \ref{case:s1 neq s2 neq |s3|}) in the theorem by letting $T = I_{3 \times 3}$.
	
	For (iii), a similar calculation as in \eqref{eqn:MFG-equivalent-Miuc-MFGI} and \eqref{eqn:MFG-equivalent-Miuc-MFGB} shows $\tilde{\mu}_c = \tilde{\mu} + \tilde{P}T^T \nu_R$.
	The sufficiency of the conditions on $\mu$ and $P$ in case \ref{case:s1=s2=-s3 neq 0}) and \ref{case:s1 neq s2=-s3 neq 0}) is clear.
	For necessity, substituting $Q=I_{3\times 3}$ proves $\mu=\tilde{\mu}$; substituting $Q=\begin{bmatrix} -1 & 0 & 0 \\ 0 & 0 & 1 \\ 0 & 1 & 0 \end{bmatrix}$, $Q=\begin{bmatrix} 0 & 0 & 1 \\ 0 & -1 & 0 \\ 1 & 0 & 0 \end{bmatrix}$, and $Q=\expb{\tfrac{\pi}{2}\hat{e}_3}$ proves $\tilde{P} = PT$, because $s_2-s_3>0$, $s_1-s_3>0$, and $s_1+s_2>0$ in these two cases.
	Finally, $\tilde{\Sigma}_c$ is
	\begin{equation}
		\tilde{\Sigma}_c = \tilde{\Sigma} + PT(\tr{S}I-S)T^TP^T.
	\end{equation}
	However, different from (ii), $T(\tr{S}I_{3 \times 3}-S)T^T \neq \tr{S}I_{3 \times 3}-S$.
	And as a result, $\Sigma_c = \tilde{\Sigma}_c$ if and only if $\tilde{\Sigma} = \Sigma + P\left[ T(\tr{S}I-S)T^T - (\tr{S}I-S) \right]P^T$.
\end{proof}

In other words, Theorem \ref{thm:MFG-equivalent} states that if $F$ has repeated singular values and $s_3\geq 0$, then MFG can be parameterized differently by rotating $U$, $V$, and $P$ in a consistent way.
This is intuitively straightforward by looking at how the correlation $P$ is constructed in Theorem \ref{thm:MFG-construction}.
When $s_3<0$, it is noticeable that $\Sigma$ and $\tilde{\Sigma}$ are also different as indicated in case \ref{case:s1=s2=-s3 neq 0}) and \ref{case:s1 neq s2=-s3 neq 0}).
The reason for this is when $s_2=-s_3$, $T(\tr{S}I_{3\times 3}-S)T^T \neq \tr{S}I_{3\times 3}-S$.

\chapter{Deterministic Attitude Observability} \label{app:obervability-deterministic}

This appendix presents parallel results for Theorem \ref{thm:observability-nonobs} and Theorem \ref{thm:observability-obs} under deterministic settings.
Specifically, the attitude kinematics \eqref{eqn:observability-kinematics-right} and \eqref{eqn:observability-kinematics-left} are formulated without noise as
\begin{align}
	\dot{R} &= \hat{\omega}(t) R, \label{eqn:observability-kinematics-right-deterministic} \\
	\dot{R} &= R\hat{\Omega}(t). \label{eqn:observability-kinematics-left-deterministic}
\end{align}
Similarly, the single reference direction \eqref{eqn:observability-measurement-inertial} and \eqref{eqn:observability-measurement-body} are also measured without noise:
\begin{align}
	x &= R^Ta, \label{eqn:observability-measurement-inertial-deterministic} \\
	y &= Rb. \label{eqn:observability-measurement-body-deterministic}
\end{align}
Then the discrete version of Theorem \ref{thm:observability-nonobs} and Theorem \ref{thm:observability-obs} is formulated as \cite{hermann1977nonlinear}:

\begin{theorem}
	The system \eqref{eqn:observability-kinematics-right-deterministic} and \eqref{eqn:observability-measurement-inertial-deterministic}, and the system \eqref{eqn:observability-kinematics-left-deterministic} and \eqref{eqn:observability-measurement-body-deterministic} are weakly locally observable.
	Conversely, the system \eqref{eqn:observability-kinematics-right-deterministic} and \eqref{eqn:observability-measurement-body-deterministic}, and the system \eqref{eqn:observability-kinematics-left-deterministic} and \eqref{eqn:observability-measurement-inertial-deterministic} are unobservable.
\end{theorem}
\begin{proof}
	The proof is based on the results of \cite{hermann1977nonlinear}, and we adopt notations therein without reintroducing them here for brevity.
	Without loss of generality, we assume $a = b = e_1$.

	For \eqref{eqn:observability-kinematics-right-deterministic}, $\mathcal{F}^0(R)$ is spanned by $\{\hat e_i R\}_{i=1}^3$.
	For any $R\in\SO{3}$ and $\hat{\eta}\in\so{3}$, the Lie derivative of $x(R)$ along $\hat{\eta}R$ is
	\begin{align*}
		(L_{(\hat{\eta}R)}x)(R) = \frac{\diff}{\diff t} \Big\lvert_{t=0} 	R^T\exp(t\hat{\eta})^Ta = R^T\hat{\eta}^Ta.
	\end{align*}
	Let $\tilde{\mathcal{G}} = \{L_{(\hat{e}_2R)}x, L_{(\hat{e}_3R)}x\} \subset \mathcal{G}$.
	For any $R_0\in\SO{3}$, define a local coordinate $\theta\in\mathbb{R}^3$ with $R(\theta) = \exp(\hat\theta)R_0$.
	Then $\diff R = \widehat{\diff \theta}R_0$, and it can be shown that
	\begin{align*}
		\diff(L_{(\hat{e}_2R_0)}x)(R_0) &= R_0^T \widehat{\diff\theta}^T \hat e_2^T e_1 = R_0^T \hat e_3 \diff\theta, \\
		\diff(L_{(\hat{e}_3R_0)}x)(R_0) &= R_0^T \widehat{\diff\theta}^T \hat e_3^T e_1  =-R_0^T \hat e_2 \diff\theta.
	\end{align*}
	Thus, $\diff \tilde{\mathcal{G}}(R_0) = R_0^T [\hat e_3, -\hat e_2]d\theta$.
	Since $\mathrm{rank}[\hat e_3,-\hat e_2]=3$, it follows that the dimension of $\diff\mathcal{G}(R_0)$ is three.
	Therefore, the system \eqref{eqn:observability-kinematics-right-deterministic} and \eqref{eqn:observability-measurement-inertial-deterministic} is weakly locally observable~\cite[Theorem 3.2]{hermann1977nonlinear}.

	Next, for any $R\in\SO{3}$ and $\hat{\eta}_1, \hat{\eta}_2\in\so{3}$, it can be shown that
	\begin{align*}
		\diff \big(L_{(\hat{\eta}_1R)}(L_{(\hat{\eta}_2R)}y)\big)(R_0) = \hat{\eta}_1\hat{\eta}_2\widehat{\diff \theta}R_0b = -\hat{\eta}_1\hat{\eta}_2 \widehat{R_0b} \diff \theta.
	\end{align*}
	As such, any higher-order Lie derivative of $y$ along \eqref{eqn:observability-kinematics-right-deterministic} would include the factor $\widehat{R_0 b}$, which has rank two.
	Thus, the dimension of $\diff \mathcal{G}(R_0)$ is at most two.
	Because the system \eqref{eqn:observability-kinematics-right-deterministic} is locally controllable, the system of \eqref{eqn:observability-kinematics-right-deterministic} and \eqref{eqn:observability-measurement-body-deterministic} is unobservable according to~\cite[Theorem 3.12]{hermann1977nonlinear}.
	The remaining cases with \eqref{eqn:observability-kinematics-left-deterministic} can be shown similarly.
\end{proof}

\chapter{Proof for Theorem \ref{thm:attEst-prop-otherMoments}} \label{app:attEst-prop-otherMoments}

\begin{proof}[Proof of Theorem \ref{thm:attEst-prop-otherMoments}]
	Equation \eqref{eqn:attEst-prop-Ex_{k+1}} and \eqref{eqn:attEst-prop-Exx_{k+1}} follow from the independence of $x_k$ and $\Delta W_v$, and the linearity of expectation.
	Equation \eqref{eqn:attEst-prop-EvR_{k+1}} is from the fact that $\expect{Q_{k+1}} = U_{k+1}^T\expect{R_{k+1}}V_{k+1}$ and $S_{k+1}$ are both diagonal.
	Next, we prove \eqref{eqn:attEst-prop-ExvR_{k+1}} and \eqref{eqn:attEst-prop-EvRvR_{k+1}}.
	
	In \eqref{eqn:attEst-kinematics-att-factorization}, denote the exponent of the stochastic part as $\xi \triangleq h(x_k-\mu_k) + H_u\Delta W_u\in\mathbb{R}^3$, and the deterministic part as $\delta R_k \triangleq e^{h(\hat{\Omega}_k+\hat{\mu}_k)}\in\SO{3}$ such that
	\begin{align*}
		R_{k+1} = R_k e^{\xi + o(h)} \delta R_k.
	\end{align*}
	First consider the MFGI definition.
	For any $A \in \mathbb{R}^{3\times 3}$, let $\Lambda(A)\in\mathbb{R}^3$ be defined as $\Lambda(A) = (U_{k+1}^TAV_{k+1}S_{k+1}-S_{k+1}V_{k+1}^TA^TU_{k+1})^\vee$. 
	For example, $\nu_{R_{k+1}} = \Lambda(R_{k+1})$.
	Further, $\Delta W_v$ in \eqref{eqn:attEst-kinematics-bias-dist} is independent of $R_k$ and $\xi$.
	Therefore, 
	\begin{align*}
		\expect{x_{k+1}\nu^T_{R_{k+1}}} = \sum_{i=0}^\infty \frac{1}{i!} \expect{x_k \Lambda (R_k(\hat{\xi}+o(h))^i \delta R_k)^T},
	\end{align*}
	after expanding $e^{\xi + o(h)}$. 
	Similar with \eqref{eqn:attEst-prop-E(R_{k+1})-taylor},  the first order approximation is 
	\begin{align} \label{eqn:attEst-prop-ExvR_{k+1}-taylor}
		\expect{x_{k+1}\nu^T_{R_{k+1}}} &= \expect{x_k \Lambda(R_k\delta R_k )^T} + h\expect{x_k \Lambda( R_k(\hat x_k-\hat{\mu}_k)\delta R_k )^T} \nonumber \\
		&\qquad + \tfrac{1}{2}\expect{x_k \Lambda( R_k((H_u\Delta W_u)^\wedge)^2\delta R_k )^T} + O(h^2).
	\end{align}
	Now we calculate the three terms on the right hand side of \eqref{eqn:attEst-prop-ExvR_{k+1}-taylor}.
	Similar with \eqref{eqn:MFG-Exx-proof}, the first term is
	\begin{align*}
		\expect{x_k\Lambda( R_k\delta R_k )^T} & = \mu_k\expect{\Lambda( R_k\delta R_k )}^T + P_k \expect{\nu_{R_k} \Lambda(R_k\delta R_k)^T },
	\end{align*}
	where
	\begin{align*}
		&\expect{\Lambda( R_k\delta R_k )} = \tilde{U} (\expect{Q_k}\tilde{S}^T-\tilde{S}\expect{Q_k}^T)^\vee = \tilde{U}\expect{\tilde{\nu}_R}, \\
		&\expect{\nu_{R_k} \Lambda( R_k\delta R )^T} = \expect{(Q_k S_k-S_kQ_k^T)^\vee ( (Q_k\tilde{S}^T-\tilde{S}Q_k^T)^\vee )^T} \tilde{U}^T = \expect{\nu_{R_k} \tilde{\nu}^T_R}\tilde{U}^T.
	\end{align*}
	Next, it can be shown that
	\begin{align*}
		\Lambda ( R_k(\hat{x}_k-\hat{\mu}_k)\delta R_k ) = \tilde{U}\left( \mathrm{tr}(Q_k\tilde{S}^T)I - Q_k\tilde{S}^T \right)Q_kV_k^T(x_k-\mu_k) = \tilde{U}\Gamma_QV_k^T(x_k-\mu_k).
	\end{align*}
	Therefore, the second term of \eqref{eqn:attEst-prop-ExvR_{k+1}-taylor} is
	\begin{align*}
		\expect{x_k \Lambda ( R_k(\hat{x}_k-\hat{\mu}_k)\delta R_k )^T} &= \expect{x_k(x_k-\mu_k)^TV_k\Gamma_Q^T}\tilde{U}^T \\
		&= \expect{(\Sigma_{c_k} + \mu_k\nu_{R_k}^TP_k^T + P_k\nu_{R_k}\nu_{R_k}^TP_k^T)V_k \Gamma_Q^T}\tilde{U}^T.
	\end{align*}
	Due to the independence of $\Delta W_u$ with other random variables, the third term of \eqref{eqn:attEst-prop-ExvR_{k+1}-taylor} is given by
	\begin{align*}
		\expect{x_k \Lambda( R_k((H_u\Delta W_u)^\wedge)^2\delta R_k )^T} = h\expect{x_k\Lambda( R_kG_u\delta R_k )^T} - h\tr{G_u}\expect{x_k\Lambda( R_k\delta R_k )^T}.
	\end{align*}
	Let $\tilde{\tilde{V}} = V_{k+1}^T\delta R_k^T G_u^TV_k$ and $\tilde{\tilde{S}} = \tilde{U}^TS_k\tilde{\tilde{V}}$. 
	The first term of the above is
	\begin{align*}
		&\expect{x_k \Lambda( R_kG_u\delta R_k )^T} = \mu\left( (\expect{Q_k}\tilde{\tilde{S}}^T-\tilde{\tilde{S}}\expect{Q_k}^T)^\vee \right)^T \tilde{U}^T \\
		&\qquad\qquad + P_k\expect{ \nu_{R_k} \left( (Q\tilde{\tilde{S}}^T-\tilde{\tilde{S}}Q^T)^\vee \right)^T}\tilde{U}^T
		= \mu\expect{\tilde{\tilde{\nu}}^T_R}\tilde{U}^T + P\expect{\nu_R\tilde{\tilde{\nu}}^T_R}\tilde{U}^T.
	\end{align*}
	Substituting these three terms into \eqref{eqn:attEst-prop-ExvR_{k+1}-taylor}, equation \eqref{eqn:attEst-prop-ExvR_{k+1}-MFGI} is derived.
	The calculation for MFGB \eqref{eqn:attEst-prop-ExvR_{k+1}-MFGB} is similar, and it is omitted for brevity.
	But it should be noted that $\tilde{\tilde{V}} \notin \SO{3}$, so \eqref{eqn:attEst-prop-vRTT-MFGB} cannot be written into a compact form like \eqref{eqn:attEst-prop-vRTT-MFGI}.
	
	Finally, we present the proof for \eqref{eqn:attEst-prop-EvRvR_{k+1}}.
	Similar to the derivation of \eqref{eqn:attEst-prop-ExvR_{k+1}-taylor}, after expanding the exponential term, the following equation can be obtained
	\begin{align*}   
		\expect{v_{R_{k+1}}v^T_{R_{k+1}}} = \sum_{i,j=0}^\infty \frac{1}{i! j!} \expect{\Lambda( R_k(\hat{\xi}+o(h))^i \delta R_k) \Lambda( R_k(\hat{\xi}+o(h))^j \delta R_k)^T}.
	\end{align*}
	After collecting the first order terms, it becomes
	\begin{align} \label{eqn:attEst-prop-EvRvR_{k+1}-taylor}
		\expect{v_{R_{k+1}}v^T_{R_{k+1}}} &= \expect{ \Lambda( R_k\delta R )\Lambda( R_k\delta R )^T } + h\expect{\Lambda( R_k\delta R )\, \Lambda( R_k(\hat{x}_k-\hat{\mu}_k)\delta R_k )^T} \nonumber \\
		&\qquad + h\expect{\Lambda( R_k\delta R )\, \Lambda( R_k(\hat{x}_k-\hat{\mu}_k)\delta R_k )^T}^T \nonumber \\
		&\qquad + \tfrac{1}{2}\expect{\Lambda( R_k\delta R )\, \Lambda( R_k((H_u\Delta W_u)^\wedge)^2\delta R_k )^T} \nonumber \\
		&\qquad + \tfrac{1}{2}\expect{\Lambda( R_k\delta R )\, \Lambda( R_k((H_u\Delta W_u)^\wedge)^2\delta R_k )^T}^T \nonumber \\
		&\qquad + \expect{\Lambda( R_k(H_u\Delta W_u)^\wedge\delta R_k )\, \Lambda( R_k(H_u\Delta W_u)^\wedge\delta R_k )^T } + O(h^2), 
	\end{align}
	where the first term on the right hand side is
	\begin{align*}
		\expect{\Lambda( R_k\delta R )\, \Lambda( R_k\delta R )^T} = \tilde{U} \expect{(Q_k\tilde{S}^T-\tilde{S}Q_k)^\vee((Q_k \tilde{S}^T-\tilde{S}Q_k)^\vee)^T} \tilde{U}^T = \tilde{U}\expect{\tilde{\nu}_R\tilde{\nu}^T_R}\tilde{U}^T,
	\end{align*}
	and the second term is 
	\begin{align*}
		\expect{\Lambda( R_k\delta R_k )\, \Lambda( R_k(\hat{x}_k-\hat{\mu}_k )\delta R_k )^T} &= \tilde{U} \expect{(Q_k \tilde{S}^T-\tilde{S}Q_k^T)^\vee \nu^T_{R_k} P_k^T V_k \Gamma_Q^T} \tilde{U}^T \\
		&= \tilde{U} \expect{\tilde{\nu}_R\nu_{R_k}^T P^T V_k \Gamma_Q^T} \tilde{U}^T,
	\end{align*}
	Similarly, the fourth term on the right hand side of \eqref{eqn:attEst-prop-EvRvR_{k+1}-taylor} is
	\begin{align*}
		\expect{\Lambda( R_k\delta R_k ) \Lambda( R_k((H_u\Delta W_u)^\wedge)^2\delta R_k )^T } = h\tilde{U}\expect{\tilde{\nu}_R\tilde{\tilde{\nu}}^T_R}\tilde{U}^T - \tr{G_u}h\tilde{U}\expect{\tilde{\nu}_R\tilde{\nu}_R^T}\tilde{U}^T,
	\end{align*}
	and the sixth term is
	\begin{align*}
		\expect{\Lambda( R_k(H_u\Delta W_u)^\wedge\delta R_k ) \Lambda( R_k(H_u\Delta W_u)^\wedge\delta R_k )^T} = h\tilde{U} \expect{\Gamma_Q V_k^T G_u V_k \Gamma_Q^T} \tilde{U}^T.
	\end{align*}
	Substituting these into \eqref{eqn:attEst-prop-EvRvR_{k+1}-taylor} yields \eqref{eqn:attEst-prop-EvRvR_{k+1}-MFGI}.
	The calculation for MFGB \eqref{eqn:attEst-prop-EvRvR_{k+1}-MFGB} is similar, and it is omitted for brevity.
\end{proof}

Note that the expression for $\Gamma_Q$ of MFGB \eqref{eqn:attEst-prop-GammaQ-MFGB} only has the first order of $Q$, but the expression for $\Gamma_Q$ of MFGI \eqref{eqn:attEst-prop-GammaQ-MFGI} has the second order of $Q$.
Fortunately, it is shown in the next lemma that \eqref{eqn:attEst-prop-GammaQ-MFGI} can also be reduced to the first order of $Q$, so that the right hand side of \eqref{eqn:attEst-prop-ExvR_{k+1}-MFGI} and \eqref{eqn:attEst-prop-EvRvR_{k+1}-MFGI} only involve at most the third order moments of $Q$.

\begin{lemma} \label{lemma:attEst-GammaQ}
	$\Gamma_Q$ in \eqref{eqn:attEst-prop-GammaQ-MFGI} can be simplified as follows:
	\begin{align}
		(\Gamma_Q)_{11} &= \tilde{S}_{22}Q_{33} + \tilde{S}_{33}Q_{22} - \tilde{S}_{23}Q_{32} - \tilde{S}_{32}Q_{23} \nonumber \\
		(\Gamma_Q)_{12} &= \tilde{S}_{23}Q_{31} + \tilde{S}_{31}Q_{23} - \tilde{S}_{21}Q_{33} - \tilde{S}_{33}Q_{21} \nonumber \\
		(\Gamma_Q)_{13} &= \tilde{S}_{21}Q_{32} + \tilde{S}_{32}Q_{21} - \tilde{S}_{22}Q_{31} - \tilde{S}_{31}Q_{22} \nonumber \\
		(\Gamma_Q)_{21} &= \tilde{S}_{13}Q_{32} + \tilde{S}_{32}Q_{13} - \tilde{S}_{12}Q_{33} - \tilde{S}_{33}Q_{12} \nonumber \\
		(\Gamma_Q)_{22} &= \tilde{S}_{11}Q_{33} + \tilde{S}_{33}Q_{11} - \tilde{S}_{13}Q_{31} - \tilde{S}_{31}Q_{13} \nonumber \\
		(\Gamma_Q)_{23} &= \tilde{S}_{12}Q_{31} + \tilde{S}_{31}Q_{12} - \tilde{S}_{11}Q_{32} - \tilde{S}_{32}Q_{11} \nonumber \\
		(\Gamma_Q)_{31} &= \tilde{S}_{12}Q_{23} + \tilde{S}_{23}Q_{12} - \tilde{S}_{13}Q_{22} - \tilde{S}_{22}Q_{13} \nonumber \\
		(\Gamma_Q)_{32} &= \tilde{S}_{13}Q_{21} + \tilde{S}_{21}Q_{13} - \tilde{S}_{11}Q_{23} - \tilde{S}_{23}Q_{11} \nonumber \\
		(\Gamma_Q)_{33} &= \tilde{S}_{11}Q_{22} + \tilde{S}_{22}Q_{11} - \tilde{S}_{12}Q_{21} - \tilde{S}_{21}Q_{12}.
	\end{align}
\end{lemma}
\begin{proof}
	This is directly calculated using the fact that for any $Q\in\SO{3}$, $Q_{ij} = (Q^{-1})_{ji} =(-1)^{i+j}M_{ij}$, where $M_{ij}$ is the minor of $Q_{ij}$.
\end{proof}

\chapter{Propagating Moments for IMU Kinematics} \label{app:posEst-prop-moments}

This appendix completes the calculations of $\expect{x_{k+1}\nu_{R_{k+1}}^T}$ and $\expect{x_{k+1}x_{k+1}^T}$ in Chapter \ref{section:posEst-propagation}, which are propagated from $(R_k,x_k) \sim \mathcal{MG}(\mu_k,\Sigma_k,P_k,U_k,S_k,V_k)$ through the IMU kinematic equations \eqref{eqn:posEst-kinematics-attitude}-\eqref{eqn:posEst-kinematics-accebias}.

The notation $\Lambda$ in the proof of \eqref{thm:attEst-prop-otherMoments} is used in this appendix.
Specifically, for any $A\in\mathbb{R}^{3\times 3}$, it is defined as
\begin{flalign*}
	\text{(MFGI)} && \Lambda(A) = \left( U_{k+1}^TAV_{k+1}S_{k+1} - S_{k+1}V_{k+1}^TA^TU_{k+1} \right)^\vee, && \\
	\text{(MFGB)} && \Lambda(A) = \left( S_{k+1}U_{k+1}^TAV_{k+1} - V_{k+1}^TA^TU_{k+1}S_{k+1} \right)^\vee. &&
\end{flalign*}

\begin{theorem} \label{thm:posEst-prop-ExvR}
	The moment $\expect{x_{k+1}\nu_{R_{k+1}}^T}$ is calculated in the following equation:
	\begin{align} \label{eqn:posEst-prop-ExvR_{k+1}-taylor}
		\expect{x_{k+1}\nu_{R_{k+1}}^T} &= \left( 1 - \tfrac{\tr{G_{gu}}h}{2} \right) \expect{x_{k+1}\Lambda\left(R_k\delta R_k\right)^T} + h\expect{x_{k+1}\Lambda\left( R_k (\hat{b}_{g,k}-\hat{\mu}_{b_g,k}) \delta R_k \right)^T} \nonumber \\
		&\qquad + \tfrac{h}{2} \expect{x_{k+1}\Lambda\left( R_kG_{gu}\delta R_k \right)^T} + O(h^2),
	\end{align}
	where $\delta R_k = \expb{h(\hat{\Omega}_k + \hat{\mu}_{b_g,k})}$.
	Define $\tilde{U}, \tilde{V}\in\SO{3}$, $\tilde{S}, \tilde{\tilde{V}}, \tilde{\tilde{S}}\in\mathbb{R}^{3\times 3}$, $\tilde{\nu}, \tilde{\tilde{\nu}}\in\mathbb{R}^3$ and $\Gamma_Q\in\mathbb{R}^{3\times 3}$ as in Theorem \ref{thm:attEst-prop-otherMoments}.
	Then, the first term $\expect{x_{k+1}\Lambda\left(R_k\delta R_k\right)^T}$ in \eqref{eqn:posEst-prop-ExvR_{k+1}-taylor} can be decomposed into blocks $\expect{b_{g,k+1}\Lambda\left(R_k\delta R_k\right)^T}$, $\expect{p_{k+1}\Lambda\left(R_k\delta R_k\right)^T}$, $\expect{v_{k+1}\Lambda\left(R_k\delta R_k\right)^T}$, and $\expect{b_{a,k+1}\Lambda\left(R_k\delta R_k\right)^T}$.
	The moment $\expect{b_{g,k+1}\Lambda\left(R_k\delta R_k\right)^T}$ is
	\begin{subequations} \label{eqn:posEst-prop-EbgLambda}
		\allowdisplaybreaks
		\begin{flalign}
			\text{(MFGI)} && \expect{b_{g,k+1}\Lambda\left(R_k\delta R_k\right)^T} &= \left( \mu_{b_g,k}\expect{\tilde{\nu}_R}^T + P_{b_g,k}\expect{\nu_{R_k}\tilde{\nu}_R^T} \right) \tilde{U}^T, && \\
			\text{(MFGB)} && \expect{b_{g,k+1}\Lambda\left(R_k\delta R_k\right)^T} &= \left( \mu_{b_g,k}\expect{\tilde{\nu}_R}^T + P_{b_g,k}\expect{\nu_{R_k}\tilde{\nu}_R^T} \right) \tilde{V}^T. &&
		\end{flalign}
	\end{subequations}
	The moment $\expect{p_{k+1}\Lambda\left(R_k\delta R_k\right)^T}$ is
	\begin{subequations} \label{eqn:posEst-prop-EpLambda}
		\allowdisplaybreaks
		\begin{flalign}
			\text{(MFGI)} && \expect{p_{k+1}\Lambda\left(R_k\delta R_k\right)^T} &= \left( \mu_{p,k}\expect{\tilde{\nu}_R}^T + P_{p,k}\expect{\nu_{R_k}\tilde{\nu}_R^T} \right) \tilde{U}^T && \nonumber \\
			&& &\qquad + h\left( \mu_{v,k}\expect{\tilde{\nu}_R}^T + P_{v,k}\expect{\nu_{R_k}\tilde{\nu}_R^T} \right) \tilde{U}^T, && \\
			\text{(MFGB)} && \expect{p_{k+1}\Lambda\left(R_k\delta R_k\right)^T} &= \left( \mu_{p,k}\expect{\tilde{\nu}_R}^T + P_{p,k}\expect{\nu_{R_k}\tilde{\nu}_R^T} \right) \tilde{V}^T && \nonumber \\
			&& &\qquad + h\left( \mu_{v,k}\expect{\tilde{\nu}_R}^T + P_{v,k}\expect{\nu_{R_k}\tilde{\nu}_R^T} \right) \tilde{V}^T. &&
		\end{flalign}
	\end{subequations}
	The moment $\expect{v_{k+1}\Lambda\left(R_k\delta R_k\right)^T}$ is
	\begin{subequations} \label{eqn:posEst-prop-EvLambda}
		\allowdisplaybreaks
		\begin{flalign}
			\text{(MFGI)} && &\expect{v_{k+1}\Lambda\left(R_k\delta R_k\right)^T} = \left( \mu_{v,k}\expect{\tilde{\nu}_R}^T + P_{v,k}\expect{\nu_{R_k}\tilde{\nu}_R^T} - hg\expect{\tilde{\nu}_R}^T \right) \tilde{U}^T && \nonumber \\
			&& &\qquad\qquad + h\left( \expect{R_k(a_k+\mu_{b_a,k})\tilde{\nu}_R^T} + \expect{R_kP_{b_a,k}\nu_{R_k}\tilde{\nu}_R^T} \right) \tilde{U}^T, && \\
			\text{(MFGB)} && &\expect{v_{k+1}\Lambda\left(R_k\delta R_k\right)^T} = \left( \mu_{v,k}\expect{\tilde{\nu}_R}^T + P_{v,k}\expect{\nu_{R_k}\tilde{\nu}_R^T} - hg\expect{\tilde{\nu}_R}^T \right) \tilde{V}^T && \nonumber \\
			&& &\qquad\qquad + h\left( \expect{R_k(a_k+\mu_{b_a,k})\tilde{\nu}_R^T} + \expect{R_kP_{b_a,k}\nu_{R_k}\tilde{\nu}_R^T} \right) \tilde{V}^T
		\end{flalign}
	\end{subequations}
	And the moment $\expect{b_{a,k+1}\Lambda\left(R_k\delta R_k\right)^T}$ is
	\begin{subequations} \label{eqn:posEst-prop-EbaLambda}
		\allowdisplaybreaks
		\begin{flalign}
			\text{(MFGI)} && \expect{b_{a,k+1}\Lambda\left(R_k\delta R_k\right)^T} &= \left( \mu_{a_g,k}\expect{\tilde{\nu}_R}^T + P_{a_g,k}\expect{\nu_{R_k}\tilde{\nu}_R^T} \right) \tilde{U}^T, && \\
			\text{(MFGB)} && \expect{b_{a,k+1}\Lambda\left(R_k\delta R_k\right)^T} &= \left( \mu_{a_g,k}\expect{\tilde{\nu}_R}^T + P_{a_g,k}\expect{\nu_{R_k}\tilde{\nu}_R^T} \right) \tilde{V}^T. &&
		\end{flalign}
	\end{subequations}
	
	Next, the second term $\expect{x_{k+1}\Lambda\left( R_k (\hat{b}_{g,k}-\hat{\mu}_{b_g,k}) \delta R_k \right)^T}$ in \eqref{eqn:posEst-prop-ExvR_{k+1}-taylor} can be approximated by
	\begin{subequations} \label{eqn:posEst-prop-ExLambda2}
		\allowdisplaybreaks
		\begin{flalign}
			\text{(MFGI)} && &\expect{x_{k+1}\Lambda\left( R_k (\hat{b}_{g,k}-\hat{\mu}_{b_g,k}) \delta R_k \right)^T} = \Big( \Sigma_{c,xb_g,k}V_k\expect{\Gamma_Q^T} && \nonumber \\
			&& &\qquad + \mu_k\expect{\nu_{R_k}^TP_{b_g,k}^TV_k\Gamma_Q^T} + P_k\expect{\nu_{R_k}\nu_{R_k}^TP_{b_g,k}^TV_k\Gamma_Q^T} \Big) \tilde{U}^T + O(h), && \\
			\text{(MFGB)} && &\expect{x_{k+1}\Lambda\left( R_k (\hat{b}_{g,k}-\hat{\mu}_{b_g,k}) \delta R_k \right)^T} = \Big( \Sigma_{c,xb_g,k}V_k\expect{\Gamma_Q^T} && \nonumber \\
			&& &\qquad + \mu_k\expect{\nu_{R_k}^TP_{b_g,k}^TV_k\Gamma_Q^T} + P_k\expect{\nu_{R_k}\nu_{R_k}^TP_{b_g,k}^TV_k\Gamma_Q^T} \Big) \tilde{V}^T + O(h), &&
		\end{flalign}
	\end{subequations}
	where $\Sigma_{c,xb_g,k}$ is the first three columns of $\Sigma_c$ of $\mathcal{MG}(\mu_k,\Sigma_k,P_k,U_k,S_k,V_k)$.
	
	Finally, the third term $\expect{x_{k+1}\Lambda\left( R_kG_{gu}\delta R_k \right)^T}$ in \eqref{eqn:posEst-prop-ExvR_{k+1}-taylor} is
	\begin{subequations} \label{eqn:posEst-prop-ExLambda3}
		\allowdisplaybreaks
		\begin{flalign}
			\text{(MFGI)} && \expect{x_{k+1}\Lambda\left( R_kG_{gu}\delta R_k \right)^T} &= \left( \mu_k\expect{\tilde{\tilde{\nu}}_R}^T + P_k\expect{\nu_{R_k}\tilde{\tilde{\nu}}_R^T} \right) \tilde{U}^T + O(h), && \\
			\text{(MFGB)} && \expect{x_{k+1}\Lambda\left( R_kG_{gu}\delta R_k \right)^T} &= \mu_k\expect{\tilde{\tilde{\nu}}_R}^T + P_k\expect{\nu_{R_k}\tilde{\tilde{\nu}}_R^T} + O(h) &&
		\end{flalign}
	\end{subequations}
\end{theorem}
\begin{proof}
	Equation \eqref{eqn:posEst-prop-ExvR_{k+1}-taylor} can be derived exactly the same as \eqref{eqn:attEst-prop-ExvR_{k+1}-taylor} in the proof of Theorem \ref{thm:attEst-prop-E(R_{k+1})}, and that $\Delta W_{gu}$ is independent on $(R_k,x_k)$, and $\expect{((H_{gu}\Delta W_{gu})^\wedge)^2} = h(G_u- \tr{G_u}I_{3\times 3})$.
	Note that $\Lambda(R_k\delta R_k) = \tilde{U}\tilde{\nu}_R$ for MFGI, and $\Lambda(R_k\delta R_k) = \tilde{V}\tilde{\nu}_R$ for MFGB.
	Using a similar calculation as \eqref{eqn:MFG-Exx-proof}, it can be shown that
	\begin{align} \label{eqn:posEst-prop-ExLambda1-proof}
		\expect{x_k\tilde{\nu}_R^T} = \mu_k\expect{\tilde{\nu}_R}^T + P_k\expect{\nu_{R_k}\tilde{\nu}_R^T}.
	\end{align}
	Then \eqref{eqn:posEst-prop-EbgLambda}, \eqref{eqn:posEst-prop-EpLambda}, and \eqref{eqn:posEst-prop-EbaLambda} are immediate from \eqref{eqn:posEst-kinematics-gyrobias}, \eqref{eqn:posEst-kinematics-pos}, and \eqref{eqn:posEst-kinematics-accebias}.
	For \eqref{eqn:posEst-prop-EvLambda}, it can be shown that
	\begin{align*}
		\expect{R_k(a_k+b_{a,k})\tilde{\nu}^T} = \expect{R_k(a_k+\mu_{b_a,k})\tilde{\nu}^T} + \expect{R_kP_{b_a,k}\nu_{R_k}\tilde{\nu}^T},
	\end{align*}
	and \eqref{eqn:posEst-prop-EvLambda} is derived.
	
	Next, since the second term in \eqref{eqn:posEst-prop-ExvR_{k+1}-taylor} already has $h$, \eqref{eqn:posEst-prop-ExLambda2} only needs to be calculated up to $O(h)$.
	Therefore, \eqref{eqn:posEst-prop-ExLambda2} can be approximated using
	\begin{align} \label{eqn:posEst-prop-ExLambda2-proof}
		\expect{x_{k+1}\Lambda\left( R_k (\hat{b}_{g,k}-\hat{\mu}_{b_g,k}) \delta R_k \right)^T} = \expect{x_k\Lambda\left( R_k (\hat{b}_{g,k}-\hat{\mu}_{b_g,k}) \delta R_k \right)^T} + O(h).
	\end{align}
	Note that $\Lambda\left( R_k (\hat{b}_{g,k}-\hat{\mu}_{b_g,k}) \delta R_k \right)$ can be simplified as
	\begin{flalign*}
		\text{MFGI} && \Lambda\left( R_k (\hat{b}_{g,k}-\hat{\mu}_{b_g,k}) \delta R_k \right) &= \tilde{U} \Gamma_Q V_k^T (b_{g,k}-\mu_{b_g,k}), && \\
		\text{MFGB} && \Lambda\left( R_k (\hat{b}_{g,k}-\hat{\mu}_{b_g,k}) \delta R_k \right) &= \tilde{V} \Gamma_Q V_k^T (b_{g,k}-\mu_{b_g,k}). &&
	\end{flalign*}
	Also, it can be shown that
	\begin{align*}
		\expect{x_k(b_{g,k}-\mu_{b_g,k})^TV_k\Gamma_Q^T} = \expect{\left( \Sigma_{c,xb_g,k} + \mu_k\nu_{R_k}^TP_{b_g}^T + P_k\nu_{R_k}\nu_{R_k}^TP_{b_g,k}^T \right) V_k\Gamma_Q^T},
	\end{align*}
	from which $\eqref{eqn:posEst-prop-ExLambda2}$ follows.
	
	Finally, the third term in \eqref{eqn:posEst-prop-ExvR_{k+1}-taylor} also has $h$, so \eqref{eqn:posEst-prop-ExLambda3} only needs to be calculated up to $O(h)$, and $x_{k+1}$ can be replaced by $x_k$ as in \eqref{eqn:posEst-prop-ExLambda2-proof}.
	Because $\Lambda(R_kG_{gu}\delta R_k) = \tilde{U}\tilde{\tilde{\nu}}_R$ for MFGI, and $\Lambda(R_kG_{gu}\delta R_k) = \tilde{\tilde{\nu}}_R$ for MFGB.
	\eqref{eqn:posEst-prop-ExLambda3} follows by the same calculation in \eqref{eqn:posEst-prop-ExLambda1-proof}.
\end{proof}

Next, the calculation for $\expect{x_{k+1}x_{k+1}^T}$ is given in the following theorem.
\begin{theorem} \label{thm:posEst-prop-Exx}
	Let $\expect{x_{k+1}x_{k+1}^T} \in \mathbb{R}^{12\times 12}$ be written in blocks as in \eqref{eqn:posEst-prop-blocks}.
	Then the blocks without $v_{k+1}$ are given by
	\begin{align}
		&\expect{b_{g,k+1}b_{g,k+1}^T} = \expect{b_{g,k}b_{g,k}^T} + hG_{gv}, \label{eqn:posEst-prop-Ebgbg} \\
		&\expect{b_{g,k+1}p_{k+1}^T} = \expect{b_{g,k}p_k^T} + h\expect{b_{g,k}v_k^T}, \\
		&\expect{b_{g,k+1}b_{a,k+1}^T} = \expect{b_{g,k}b_{a,k}^T}, \\
		&\expect{p_{k+1}p_{k+1}^T} = \expect{p_kp_k^T} + h\left(\expect{p_kv_k^T} + \expect{v_kp_k^T}\right) + h^2\expect{v_kv_k^T}, \\
		&\expect{p_{k+1}b_{a,k+1}^T} = \expect{p_kb_{a,k}^T} + h\expect{v_kb_{a,k}^T}, \\
		&\expect{b_{a,k+1}b_{a,k+1}^T} = \expect{b_{a,k}b_{a,k}^T} + hG_{av}, \label{eqn:posEst-prop-Ebaba}
	\end{align}
	where the expectations on the right hand side can be calculated using the blocks of $\expect{x_kx_k^T}$.
	Next, the block $\expect{v_{k+1}b_{g,k+1}^T}$ is given by
	\begin{align} \label{eqn:posEst-prop-Evbg}
		\expect{v_{k+1}b_{g,k+1}^T} &= \expect{v_kb_{g,k}^T} - hg\expect{b_{g,k}^T} + h\left( \expect{R_k}a_k + \expect{R_kP_{b_a,k}\nu_{R_k}} \right) \mu_{b_g,k}^T \nonumber \\
		&\qquad + h\left( \expect{R_k(a_k+\mu_{b_a,k})\nu_{R_k}^T} + \expect{R_kP_{b_a,k}\nu_{R_k}\nu_{R_k}^T} \right) P_{b_g,k}^T \nonumber \\
		&\qquad + h\expect{R_k} \left( \Sigma_{c,b_ab_g,k} + \mu_{b_a,k}\mu_{b_g,k}^T \right).
	\end{align}
	The block $\expect{v_{k+1}p_{k+1}^T}$ is
	\begin{align} \label{eqn:posEst-prop-Evp}
		&\expect{v_{k+1}p_{k+1}^T} = \expect{v_kp_k^T} + h\expect{v_kv_k^T} - hg\expect{p_k^T} - h^2g\expect{v_k^T} \nonumber \\
		&\qquad\qquad + h\left( \expect{R_k}a_k + \expect{R_kP_{b_a,k}\nu_{R_k}} \right) \mu_{p,k}^T + h\expect{R_k} \left( \Sigma_{c,b_ap,k} + \mu_{b_a,k}\mu_{p,k}^T \right) \nonumber \\
		&\qquad\qquad + h\left( \expect{R_k(a_k+\mu_{b_a,k})\nu_{R_k}^T} + \expect{R_kP_{b_a,k}\nu_{R_k}\nu_{R_k}^T} \right) P_{p,k}^T \nonumber \\
		&\qquad\qquad + h^2\left( \expect{R_k}a_k + \expect{R_kP_{b_a,k}\nu_{R_k}} \right) \mu_{v,k}^T + h\expect{R_k} \left( \Sigma_{c,b_av,k} + \mu_{b_a,k}\mu_{v,k}^T \right) \nonumber \\
		&\qquad\qquad + h^2\left( \expect{R_k(a_k+\mu_{b_a,k})\nu_{R_k}^T} + \expect{R_kP_{b_a,k}\nu_{R_k}\nu_{R_k}^T} \right) P_{v,k}^T.
	\end{align}
	The block $\expect{v_{k+1}v_{k+1}^T}$ is
	\begin{align} \label{eqn:posEst-prop-Evv}
		&\expect{v_{k+1}v_{k+1}^T} = \expect{v_kv_k^T} + h^2gg^T + h \expect{R_kG_{au}R_k^T} - h\left( g\expect{v_k}^T + \expect{v_k}g^T \right) \nonumber \\
		&\qquad\qquad - h^2g\left( \expect{R_k}(a_k+\mu_{b_a,k}) + \expect{R_kP_{b_a,k}\nu_{R_k}} \right)^T \nonumber \\
		&\qquad\qquad - h^2\left( \expect{R_k}(a_k+\mu_{b_a,k}) + \expect{R_kP_{b_a,k}\nu_{R_k}} \right)g^T \nonumber \\
		&\qquad\qquad + h\left( \expect{R_k}a_k + \expect{R_kP_{b_a,k}\nu_{R_k}} \right) \mu_{v,k}^T + h\expect{R_k} \left( \Sigma_{c,b_av,k} + \mu_{b_a,k}\mu_{v,k}^T \right) \nonumber \\
		&\qquad\qquad + h\left( \expect{R_k(a_k+\mu_{b_a,k})\nu_{R_k}^T} + \expect{R_kP_{b_a,k}\nu_{R_k}\nu_{R_k}^T} \right) P_{v,k}^T \nonumber \\
		&\qquad\qquad + h\mu_{v,k}\left( \expect{R_k}a_k + \expect{R_kP_{b_a,k}\nu_{R_k}} \right)^T + h\left( \Sigma_{c,vb_a,k} + \mu_{v,k}\mu_{b_a,k}^T \right) \expect{R_k}^T \nonumber \\
		&\qquad\qquad + hP_{v,k}\left( \expect{R_k(a_k+\mu_{b_a,k})\nu_{R_k}^T} + \expect{R_kP_{b_a,k}\nu_{R_k}\nu_{R_k}^T} \right)^T \nonumber \\
		&\qquad\qquad + h^2\expect{ R_k\left( \Sigma_{c,b_ab_a,k} + a_ka_k^T + a_k\mu_{b_a,k}^T + \mu_{b_a,k}a_k^T + \mu_{b_a,k}\mu_{b_a,k}^T \right)R_k^T } \nonumber \\
		&\qquad\qquad + h^2\expect{ R_k\left( (a_k+\mu_{b_a,k})\nu_{R_k}^TP_{b_a,k}^T + P_{b_a,k}\nu_{R_k}(a_k+\mu_{b_a,k})^T \right)R_k^T } \nonumber \\
		&\qquad\qquad + h^2\expect{R_kP_{b_a,k}\nu_{R_k}\nu_{R_k}^TP_{b_a,k}^TR_k^T}.
	\end{align}
	And finally, the block $\expect{v_{k+1}b_{a,k+1}^T}$ is
	\begin{align} \label{eqn:posEst-prop-Evba}
		\expect{v_{k+1}b_{a,k+1}^T} &= \expect{v_kb_{a,k}^T} - hg\expect{b_{a,k}^T} + h\left( \expect{R_k}a_k + \expect{R_kP_{b_a,k}\nu_{R_k}} \right) \mu_{b_a,k}^T \nonumber \\
		&\qquad + h\left( \expect{R_k(a_k+\mu_{b_a,k})\nu_{R_k}^T} + \expect{R_kP_{b_a,k}\nu_{R_k}\nu_{R_k}^T} \right) P_{b_a,k}^T \nonumber \\
		&\qquad + h\expect{R_k} \left( \Sigma_{c,b_ab_a,k} + \mu_{b_a,k}\mu_{b_a,k}^T \right).
	\end{align}
\end{theorem}
\begin{proof}
	Equation \eqref{eqn:posEst-prop-Ebgbg}-\eqref{eqn:posEst-prop-Ebaba} can be easily derived from \eqref{eqn:posEst-kinematics-attitude}-\eqref{eqn:posEst-kinematics-accebias}.
	For \eqref{eqn:posEst-prop-Evbg}, it can be calculated as
	\begin{align*}
		\expect{v_{k+1}b_{g,k+1}^T} = \expect{v_kb_{g,k}^T} + h\expect{R_ka_kb_{g,k}^T} + h\expect{R_kb_{a,k}b_{g,k}^T} - hg\expect{b_{g,k}^T},
	\end{align*}
	where the second and third terms can be expanded into
	\begin{align*}
		\expect{R_ka_kb_{g,k}^T} = \expect{R_k}a_k\mu_{b_g,k}^T + \expect{R_ka_k\nu_{R_k}^T}P_{b_g,k}^T,
	\end{align*}
	and
	\begin{align*}
		\expect{R_kb_{a,k}b_{g,k}^T} &= \expect{R_k} \left( \Sigma_{c,b_ab_g,k} + \mu_{b_a,k}\mu_{b_g,k}^T \right) + \expect{R_k\mu_{b_a,k}\nu_{R_k}^T}P_{b_g,k}^T \\
		&\qquad\qquad + \expect{R_kP_{b_a,k}\nu_{R_k}}\mu_{b_g,k}^T + \expect{R_kP_{b_a,k}\nu_{R_k}\nu_{R_k}^T}P_{b_g,k}^T.
	\end{align*}
	Then \eqref{eqn:posEst-prop-Evbg} can be proved after collecting the terms in the above two equations.
	The proofs for \eqref{eqn:posEst-prop-Evp}-\eqref{eqn:posEst-prop-Evba} are similar but extremely tedious, and they are omitted for brevity.
\end{proof}

Note that the term $\expect{R_kP_{b_a,k}\nu_{R_k}\nu_{R_k}^TP_{b_a,k}^TR_k^T}$ in \eqref{eqn:posEst-prop-Evv} needs the fourth order moments of $\expect{Q_k}$, which do not have closed form formula.
But its accuracy is on the order of $O(h^2)$, and can be omitted if the sampling period $h$ is small enough.

\chapter{The Posterior Density With Map Uncertainty} \label{app:VIO-map-posterior}

This appendix deals with the calculations related to the posterior density function \eqref{eqn:VIO-map-posterior}.
First, the proof for Theorem \ref{thm:VIO-map-posterior} is given.

\begin{proof}[Proof of Theorem \ref{thm:VIO-map-posterior}]
	The last two terms in \eqref{eqn:VIO-map-posterior-factor1} can be combined into
	\begin{align*}
		&(\bm{x}-\bm{\mu}_{x|R})^T \bm{\Sigma}_{x|R}^{-1} (\bm{x}-\bm{\mu}_{x|R}) + (y-\mu_c)^T \Sigma_c^{-1} (y-\mu_c) \\
		= &(Hz-\bm{\mu}_{x|R})^T \bm{\Sigma}_{x|R}^{-1} (Hz-\bm{\mu}_{x|R}) + (Gz-\mu_c)^T \Sigma_c^{-1} (Gz-\mu_c) \\
		= &z^T \left( H^T\bm{\Sigma}_{x|R}^{-1}H + G^T\Sigma_c^{-1}G \right) z - z^T \left( H^T\bm{\Sigma}_{x|R}^{-1}\mu_{x|R} + G^T\Sigma_c^{-1}\mu_c \right) + \mu_{x|R}^T \bm{\Sigma}_{x|R}^{-1} \mu_{x|R} + \mu_c^T \Sigma_c^{-1} \mu_c \\
		= & (z-\mu_{z|R})^T \Sigma_{z|R}^{-1} (z-\mu_{z|R}) - \mu_{z|R}^T \Sigma_{z|R}^{-1} \mu_{z|R} + \mu_{x|R}^T \bm{\Sigma}_{x|R}^{-1} \mu_{x|R} + \mu_c^T \Sigma_c^{-1} \mu_c.
	\end{align*}
	Substitute this into \eqref{eqn:VIO-map-posterior-factor1}, it remains to show that
	\begin{align*}
		(\mu_{t|R}-\mu_{c,t})^T (\Sigma_{t|R} + \Sigma_{c,t})^{-1} (\mu_{t|R}-\mu_{c,t}) = \mu_{x|R}^T \bm{\Sigma}_{x|R}^{-1} \mu_{x|R} + \mu_c^T \Sigma_c^{-1} \mu_c - \mu_{z|R}^T \Sigma_{z|R}^{-1} \mu_{z|R}.
	\end{align*}
	Note that $HH^T = I_{3N+3}$, and $GG^T = I_{n\times n}$, so the right hand side can be expanded into
	\begin{align} \label{eqn:VIO-three-terms}
		&\mu_{x|R}^T \bm{\Sigma}_{x|R}^{-1} \mu_{x|R} + \mu_c^T \Sigma_c^{-1} \mu_c - \mu_{z|R}^T \Sigma_{z|R}^{-1} \mu_{z|R} \nonumber \\
		= &(H^T\mu_{x|R})^T \left[ H^T\bm{\Sigma}_{x|R}^{-1}H - H^T\bm{\Sigma}_{x|R}^{-1}H \left( H^T\bm{\Sigma}_{x|R}^{-1}H + G^T\Sigma_c^{-1}G \right)^{-1} H^T\bm{\Sigma}_{x|R}^{-1}H \right] (H^T\mu_{x|R}) \nonumber \\
		&\qquad + (G^T\mu_c)^T \left[ G^T\Sigma_c^{-1}G - G^T\Sigma_c^{-1}G \left( H^T\bm{\Sigma}_{x|R}^{-1}H + G^T\Sigma_c^{-1}G \right)^{-1} G^T\Sigma_c^{-1}G \right] (G^T\mu_c) \nonumber \\
		&\qquad -2 (H^T\mu_{x|R})^T H^T\bm{\Sigma}_{x|R}^{-1} H \left( H^T\bm{\Sigma}_{x|R}^{-1}H + G^T\Sigma_c^{-1}G \right)^{-1} G^T\Sigma_c^{-1}G (G^T\mu_c).
	\end{align}
	The first term on the right hand side of \eqref{eqn:VIO-three-terms} can be simplified using block inversion formula, and Woodbury matrix identity as
	\begin{align*}
		&(H^T\mu_{x|R})^T \left[ H^T\bm{\Sigma}_{x|R}^{-1}H - H^T\bm{\Sigma}_{x|R}^{-1}H \left( H^T\bm{\Sigma}_{x|R}^{-1}H + G^T\Sigma_c^{-1}G \right)^{-1} H^T\bm{\Sigma}_{x|R}^{-1}H \right] (H^T\mu_{x|R}) \\
		= &\mu_{x|R}^T \bm{\Sigma}_{x|R}^{-1}H \left( H^T\bm{\Sigma}_{x|R}^{-1}H + G^T\Sigma_c^{-1}G \right)^{-1} G^T\Sigma_c^{-1} GH^T\mu_{x|R} \\
		= &\mu_{t|R}^T \Sigma_{t|R}^{-1} \left( \Sigma_{t|R}^{-1} + \Sigma_{c,t}^{-1} \right)^{-1} \Sigma_{c,t}^{-1} \mu_{t|R} \\
		= &\mu_{t|R}^T (\Sigma_{t|R}+\Sigma_{c,t})^{-1} \mu_{t|R},
	\end{align*}
	where the first equality is a straightforward but extremely tedious application of the block inversion formula to $\Sigma_{z|R}$, which has to be omitted here.
	And the second equality is because
	\begin{align*}
		\bm{\Sigma}_{x|R}^{-1}H \left( H^T\bm{\Sigma}_{x|R}^{-1}H + G^T\Sigma_c^{-1}G \right)^{-1} G^T\Sigma_c^{-1} = \begin{bmatrix} 0_{3N\times 3} & 0_{3N\times n} \\ \Sigma_{t|R}^{-1} \left( \Sigma_{t|R}^{-1} + \Sigma_{c,t}^{-1} \right)^{-1} \Sigma_{c,t}^{-1} & 0_{3\times n} \end{bmatrix}.
	\end{align*}
	Similarly, the second term on the right hand side of \eqref{eqn:VIO-three-terms} can be simplified as
	\begin{align*}
		&(G^T\mu_c)^T \left[ G^T\Sigma_c^{-1}G - G^T\Sigma_c^{-1}G \left( H^T\bm{\Sigma}_{x|R}^{-1}H + G^T\Sigma_c^{-1}G \right)^{-1} G^T\Sigma_c^{-1}G \right] (G^T\mu_c) \\
		= &\mu_{c,t}^T (\Sigma_{t|R}+\Sigma_{c,t})^{-1} \mu_{c,t},
	\end{align*}
	and the third term can be simplified as
	\begin{align*}
		&(H^T\mu_{x|R})^T H^T\bm{\Sigma}_{x|R}^{-1} H \left( H^T\bm{\Sigma}_{x|R}^{-1}H + G^T\Sigma_c^{-1}G \right)^{-1} G^T\Sigma_c^{-1}G (G^T\mu_c) \\
		= &\mu_{t|R} (\Sigma_{t|R}+\Sigma_{c,t})^{-1} \mu_{c,t}.
	\end{align*}
	Combining these three terms yields the desired equality.
\end{proof}

Next, the moments for $y|R$ and $p_i|R$ according to the posterior density \eqref{eqn:VIO-map-posterior-factor2} are given in the following theorem.

\begin{theorem} \label{thm:VIO-map-posterior-moments}
	Define $K_y \in \mathbb{R}^{n\times n}$ as
	\begin{align}
		K_y = \left( \begin{bmatrix} \tilde{C}^{-1} & 0_{3\times (n-3)} \\ 0_{(n-3)\times 3} & 0_{(n-3)\times (n-3)} \end{bmatrix} + \Sigma_c^{-1} \right)^{-1} \Sigma_c^{-1}.
	\end{align}
	Then the moment $\expect{y|\mathcal{Z}}$ can be calculated as
	\begin{align}
		\expect{y|\mathcal{Z}} = (I-K_y) \begin{bmatrix} \expect{\mu_{t|R}|\mathcal{Z}} \\ 0_{(n-3)\times 1} \end{bmatrix} + K_y\left( \mu + P\expect{\nu_R|\mathcal{Z}} \right).
	\end{align}
	The moment $\expect{y(\nu_R^+)^T|\mathcal{Z}}$ is
	\begin{align}
		\expect{y(\nu_R^+)^T|\mathcal{Z}} = (I-K_y) \begin{bmatrix} \expect{\mu_{t|R}(\nu_R^+)^T|\mathcal{Z}} \\ 0_{(n-3)\times 1} \end{bmatrix} + K_yP\expect{\nu_R(\nu_R^+)^T|\mathcal{Z}}.
	\end{align}
	And the moment $\expect{yy^T|\mathcal{Z}}$ is
	\begin{align} \label{eqn:VIO-map-posterior-Eyy}
		&\expect{yy^T|\mathcal{Z}} = \left( \begin{bmatrix} \tilde{C}^{-1} & 0_{3\times (n-3)} \\ 0_{(n-3)\times 3} & 0_{(n-3)\times (n-3)} \end{bmatrix} + \Sigma_c^{-1} \right)^{-1} \nonumber \\
		&\qquad + (I-K_y) \begin{bmatrix} \expect{\mu_{t|R}\mu_{t|R}^T|\mathcal{Z}} & 0_{3\times (n-3)} \\ 0_{(n-3)\times 3} & 0_{(n-3)\times (n-3)} \end{bmatrix} (I-K_y)^T \nonumber \\
		&\qquad + (I-K_y) \begin{bmatrix} \expect{\mu_{t|R}|\mathcal{Z}}\mu^T + \expect{\mu_{t|R}\nu_R^T|\mathcal{Z}}P^T \\ 0_{(n-3)\times n} \end{bmatrix} K_y^T \nonumber \\
		&\qquad + K_y \begin{bmatrix} \expect{\mu_{t|R}|\mathcal{Z}}\mu^T + \expect{\mu_{t|R}\nu_R^T|\mathcal{Z}}P^T \\ 0_{(n-3)\times n} \end{bmatrix}^T (I-K_y)^T \nonumber \\
		&\qquad + K_y\left( \mu\mu^T + \mu\expect{\nu_R^T|\mathcal{Z}} P^T + P\expect{\nu_R|\mathcal{Z}}\mu^T + P\expect{\nu_R\nu_R^T|\mathcal{Z}}P^T \right)K_y^T.
	\end{align}
	In the above three equations, $I$ is abbreviated for $I_{n\times n}$, $\expect{\nu_R|\mathcal{Z}}$, $\expect{\nu_R(\nu_R^+)^T|\mathcal{Z}}$ and $\expect{\nu_R\nu_R^T|\mathcal{Z}}$ are calculated as in Theorem \ref{thm:attEst-update-moments}.
	Also, $\expect{\mu_{t|R}|\mathcal{Z}}$, $\expect{\mu_{t|R}(\nu_R^+)^T|\mathcal{Z}}$, $\expect{\mu_{t|R}\nu_R^T|\mathcal{Z}}$, and $\expect{\mu_{t|R}\mu_{t|R}^T|\mathcal{Z}}$ are calculated as in Theorem \ref{thm:VIO-posterior-conditional-moments}, with $p_i$, $C$, $D_i$ replaced by $\bar{p}_i$, $\tilde{C}$, $\tilde{D}_i$.
	Next, the moment $\expect{p_i|\mathcal{Z}}$ is
	\begin{align}
		\expect{p_i|\mathcal{Z}} = (I_{3\times 3} - K_i) \bar{p}_i + K_i(Rp'_i + \expect{t|\mathcal{Z}}).
	\end{align}
	And the moment $\expect{p_ip_i^T|\mathcal{Z}}$ is
	\begin{align}
		&\expect{p_ip_i^T|\mathcal{Z}} = B_i - B_i\left[ (A_i+B_i)^{-1} \left( A_i + B_i + (\tilde{C}^{-1} + \Sigma_{c,t}^{-1})^{-1} \right) (A_i+B_i)^{-1} \right] B_i \nonumber \\
		&\qquad + (I_{3\times 3} - K_i) \bar{p}_i\bar{p}_i^T (I_{3\times 3} - K_i)^T + (I_{3\times 3} - K_i)\bar{p}_i \left((p'_i)^T\expect{R|\mathcal{Z}}^T + \expect{t|\mathcal{Z}}^T\right) K_i^T \nonumber \\
		&\qquad + K_i \left( \expect{Rp'_i(p'_i)^TR^T | \mathcal{Z}} + \expect{Rp'_it^T|\mathcal{Z}} + \expect{t(p'_i)^TR^T|\mathcal{Z}} + \expect{\mu'_{t|R}(\mu'_{t|R})^T|\mathcal{Z}} \right) \nonumber \\
		&\qquad + K_i \left( \expect{R|\mathcal{Z}}p'_i + \expect{t|\mathcal{Z}} \right) \bar{p}_i^T (I_{3\times 3}-K_i)^T.
	\end{align}
	In the above two equations, $\expect{t|\mathcal{Z}}$ is the first three columns of $\expect{y|\mathcal{Z}}$, and $\expect{\mu'_{t|R}(\mu'_{t|R})^T|\mathcal{Z}}$ is the first 3-by-3 diagonal block of the last four term on the right hand side of \eqref{eqn:VIO-map-posterior-Eyy}.
	Finally, let $K_t \in \mathbb{R}^{3\times 3}$ be the first 3-by-3 diagonal block of $K_y$, then
	\begin{align}
		\expect{t(p'_i)^TR^T|\mathcal{Z}} = (I_{3\times 3}-K_t) \expect{\mu_{t|R}(p'_i)^TR^T|\mathcal{Z}} + K_t \expect{\mu_{c,t}(p'_i)^TR^T|\mathcal{Z}},
	\end{align}
	where $\expect{\mu_{t|R}(p'_i)^TR^T|\mathcal{Z}}$ is given in Theorem \ref{thm:VIO-likelihood-Ep}.
\end{theorem}
\begin{proof}
	The key idea is to calculate the diagonal blocks of the covariance matrix $\Sigma_{z|R}$ and the conditional mean $\mu_{z|R}$ in Theorem \ref{thm:VIO-map-posterior}, corresponding to $y$ and $p_i$.
	First, by directly applying block inversion formula to $\Sigma_{z|R}$, the diagonal block for $y$ is calculated as
	\begin{align*}
		\Sigma'_{y|R} = \left( \begin{bmatrix} \tilde{C}^{-1} & 0_{3\times (n-3)} \\ 0_{(n-3)\times 3} & 0_{(n-3)\times (n-3)} \end{bmatrix} + \Sigma_c^{-1} \right)^{-1},
	\end{align*}
	and the block for $p_i$ is
	\begin{align*}
		\Sigma'_{p_i|R} = B_i - B_i\left[ (A_i+B_i)^{-1} \left( A_i + B_i + (\tilde{C}^{-1} + \Sigma_{c,t}^{-1})^{-1} \right) (A_i+B_i)^{-1} \right] B_i.
	\end{align*}
	Next, the conditional mean can be written as
	\begin{align*}
		\mu_{z|R} &= \left( H^T\bm{\Sigma}_{x|R}^{-1}H + G^T\Sigma_c^{-1}G \right)^{-1} \left( H^T\bm{\Sigma}_{x|R}^{-1}\mu_{x|R} + G^T\Sigma_c^{-1}\mu_c \right) \\
		&= H^T\mu_{x|R} + \left( H^T\bm{\Sigma}_{x|R}^{-1}H + G^T\Sigma_c^{-1}G \right)^{-1} \left( G^T\Sigma_c^{-1}\mu_c - G^T\Sigma_c^{-1}GH^T\mu_{x|R} \right) \nonumber \\
		&= H^T\mu_{x|R} + \left( H^T\bm{\Sigma}_{x|R}^{-1}H + G^T\Sigma_c^{-1}G \right)^{-1} G^T\Sigma_c^{-1} \left( \mu_c - \begin{bmatrix} \mu_{t|R} \\ 0_{(n-3)\times 1} \end{bmatrix} \right).
	\end{align*}
	So the component for $y$ is
	\begin{align*}
		\mu'_{y|R} = (I-K_y) \begin{bmatrix} \mu_{t|R} \\ 0_{(n-3)\times 1} \end{bmatrix} + K_y\mu_c,
	\end{align*}
	and the component for $p_i$ is
	\begin{align*}
		\mu'_{p_i|R} &= \mu_{p_i|R} + \begin{bmatrix} K_i & 0_{3\times (n-3)}  \end{bmatrix} K_y \left( \mu_c - \begin{bmatrix} \mu_{t|R} \\ 0_{(n-3)\times 1} \end{bmatrix} \right).
	\end{align*}
	Using block inversion formula, note that
	\begin{align*}
		K_y = \begin{bmatrix} \left( \tilde{C}^{-1} + \Sigma_{c,t}^{-1} \right)^{-1} \Sigma_{c,t}^{-1} & 0_{3\times n} \\ \cdot & I_{n\times n} \end{bmatrix}.
	\end{align*}
	So $\mu'_{p_i|R}$ can be simplified as
	\begin{align*}
		\mu'_{p_i|R} &= \mu_{p_i|R} + K_iK_t(\mu_{c,t}-\mu_{t|R}) \\
		&= (I-F_i)\bar{p}_i + K_i(Rp'_i+\mu_{t|R}) + K_iK_t(\mu_{c,t} - \mu_{t|R}) \\
		&= (I-F_i)\bar{p}_i + K_i(Rp'_i+\mu'_{t|R}),
	\end{align*}
	where $\mu'_{t|R}$ is the first three columns of $\mu'_{y|R}$.
	The above calculations are used when integration the left hand side of the moments with respect to the last term of \eqref{eqn:VIO-map-posterior-factor2}, and evaluating the resulting moments with respect to $\mathcal{M}(F^+)$, which is the matrix Fisher distribution matched to the first three terms on the right hand side of \eqref{eqn:VIO-map-posterior-factor2}.
\end{proof}
