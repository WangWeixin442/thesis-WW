% !TEX root = ../thesis-WW.tex
\appendix
\doublespacing

\chapter{A Special Case in Theorem \ref{thm:MF-moment-dsdT}} \label{app:MF-moment-specialRecursion}

In this supplementary material, a special case which is embedded in the recursion of Theorem \eqref{thm:MF-moment-dsdT} is untwisted into a non-recursive formula except for an integer coefficient.

\begin{theorem} \label{thm:MF-moment-specialRecursion}
	If $s_1 \neq s_2 \neq |s_3|$ and $i \neq j$, then
	\begin{align} \label{eqn:dsisjdT}
		\left.\left( \frac{\partial^{2n}s'_i}{\partial T_{ij}^{2n}} + \frac{\partial^{2n}s'_j}{\partial T_{ij}^{2n}} \right)\right|_{T=0} &= \frac{a(n)}{(s_i+s_j)^{2n-1}}, \nonumber \\
		\left.\left( \frac{\partial^{2n}s'_i}{\partial T_{ij}^{2n}} - \frac{\partial^{2n}s'_j}{\partial T_{ij}^{2n}} \right)\right|_{T=0} &= \frac{a(n)}{(s_i-s_j)^{2n-1}},
	\end{align}
	where $a(n)$ is an integer-valued function of $n$.
\end{theorem}

Before proving this theorem, a lemma for differentiating $U'$ and $V'$ is needed.
\begin{lemma} \label{lemma:MF-moment-dUVdT-zero}
	Under the conditions of Theorem \ref{thm:MF-moment-dsdT}, if any of $k$ or $l$ is distinct from $i$ or $j$, then
	\begin{equation}
		\left. \frac{\partial^n U'_{kl}}{\partial T_{ij}^n} \right|_{T=0} = 0, \qquad
		\left. \frac{\partial^n V'_{kl}}{\partial T_{ij}^n} \right|_{T=0} = 0,
	\end{equation}
	for all $n>0$.
\end{lemma}
\begin{proof}
	Suppose $i=1,j=2$, and let $\begin{bmatrix} s_1 & T_{12} \\ 0 & s_2 \end{bmatrix} = \begin{bmatrix} U'_{11} & U'_{12} \\ U'_{21} & U'_{22} \end{bmatrix} \begin{bmatrix} s'_1 & 0 \\ 0 & s'_2 \end{bmatrix} \begin{bmatrix} V'_{11} & V'_{12} \\ V'_{21} & V'_{22} \end{bmatrix}^T$ be the singular value decomposition of the upper left 2-by-2 diagonal block of $S+T$.
	Then by direct calculation, $S+T$ has singular value decomposition
	\begin{equation*}
		\begin{bmatrix}
			s_1 & T_{12} & 0 \\
			0 & s_2 & 0 \\
			0 & 0 & s_3
		\end{bmatrix} = \begin{bmatrix}
			U'_{11} & U'_{12} & 0 \\
			U'_{21} & U'_{22} & 0 \\
			0 & 0 & 1
		\end{bmatrix} \begin{bmatrix}
			s'_1 & 0 & 0 \\
			0 & s'_2 & 0 \\
			0 & 0 & s_3
		\end{bmatrix} \begin{bmatrix}
			V'_{11} & V'_{12} & 0 \\
			V'_{21} & V'_{22} & 0 \\
			0 & 0 & 1
		\end{bmatrix}^T.
	\end{equation*}
	This means if the subscripts of $U'$ or $V'$ contain $3$ which is different from $\{1,2\}$, the corresponding element of $U'$ or $V'$ is either zero or one, whose derivatives with respect to $T_{12}$ of any order evaluated at $T=0$ is zero.
	Other cases can be shown similarly.
\end{proof}

Next, the proof for Theorem \ref{thm:MF-moment-specialRecursion} is provided.
\begin{proof}
	By direct calculation using \eqref{eqn:dsvd-dsdT}, is can be shown that
	\begin{align*}
		\frac{\partial (s'_i+s'_j)}{\partial T_{ij}} &= U'_{ii}V'_{ji} + U'_{ij}V'_{jj} \triangleq X_1, \\
		\frac{\partial (s'_i-s'_j)}{\partial T_{ij}} &= U'_{ii}V'_{ji} - U'_{ij}V'_{jj} \triangleq X_2,
	\end{align*}
	for $X_1,X_2\in\mathbb{R}$.
	Next, the second order differentiation is calculated using \eqref{eqn:dsvd-dUVdT} as
	\begin{align*}
		\frac{\partial X_1}{\partial T_{ij}}
		&= (U'_{ii}V'_{jj}-U'_{ij}V'_{ji})(\Omega_{U_{ij}}^{ij}+\Omega_{V_{ij}}^{ij}) + U'_{ik}V'_{ji}\Omega_{U_{ki}}^{ij} - U'_{ii}V'_{jk}\Omega_{V_{ki}}^{ij} + U'_{ik}V'_{jj}\Omega_{U_{kj}}^{ij} - U'_{ij}V'_{jk}\Omega_{V_{kj}}^{ij}, \\
		\frac{\partial X_2}{\partial T_{ij}} &= (U'_{ii}V'_{jj}+U'_{ij}V'_{ji})(\Omega_{V_{ij}}^{ij}-\Omega_{U_{ij}}^{ij}) + U'_{ik}V'_{ji}\Omega_{U_{ki}}^{ij} - U'_{ii}V'_{jk}\Omega_{V_{ki}}^{ij} - U'_{ik}V'_{jj}\Omega_{U_{kj}}^{ij} + U'_{ij}V'_{jk}\Omega_{V_{kj}}^{ij},
	\end{align*}
	where $k\in\{1,2,3\}$ and $i\neq j\neq k$.
	The crucial observation is that if we further differentiate the above two equations with respect to $T_{ij}$ and evaluate at $T=0$, then the last four terms in each subequation yield terms involving either $U'_{ik}$, $V'_{jk}$, or their derivatives with respect to $T_{ij}$.
	However, Lemma \ref{lemma:MF-moment-dUVdT-zero} indicates that all these terms vanish after evaluated at $T=0$.
	This means when calculating \eqref{eqn:dsisjdT}, the last four terms in each of the above subequations can be simply omitted.
	Therefore, by \eqref{eqn:dsvd-OmegaUV} is can be shown that
	\begin{align*}
		\frac{\partial X_1}{\partial T_{ij}} &= (U'_{ii}V'_{jj}-U'_{ij}V'_{ji})(\Omega_{U_{ij}}^{ij}+\Omega_{V_{ij}}^{ij}) = \frac{(U'_{ii}V'_{jj}-U'_{ij}V'_{ji})^2}{s'_i+s'_j} \triangleq \frac{Y_1^2}{s'_i+s'_j}, \\
		\frac{\partial X_2}{\partial T_{ij}} &= (U'_{ii}V'_{jj}+U'_{ij}V'_{ji})(-\Omega_{U_{ij}}^{ij}+\Omega_{V_{ij}}^{ij}) = \frac{(U'_{ii}V'_{jj}+U'_{ij}V'_{ji})^2}{s'_i-s'_j} \triangleq \frac{Y_2^2}{s'_i-s'_j}.
	\end{align*}
	If the above two equations are differentiated with respect to $T_{ij}$ again, we obtain the derivatives of $s_i \pm s_j$ which have been calculated, and the derivatives of $Y_1$, $Y_2$ which are calculated as follows.
	\begin{align*}
		\frac{\partial Y_1}{\partial T_{ij}} &= -\frac{(V'_{ii}V'_{ji}+U'_{ij}V'_{jj})(U'_{ii}V'_{jj}-U'_{ij}V'_{ji})}{s'_i+s'_j} = -\frac{X_1Y_1}{s'_i+s'_j}, \\
		\frac{\partial Y_2}{\partial T_{ij}} &= -\frac{(U'_{ii}V'_{ji}-U'_{ij}V'_{ji})(U'_{ii}V'_{jj}+U'_{ij}V'_{ji})}{s'_i-s'_j} = -\frac{X_2Y_2}{s'_i-s'_j},
	\end{align*}
	where any term having the subscript $k$ has been omitted. 
	
	Now, a very special branch of smaller recursion is distilled from the large recursion stated in Theorem \ref{thm:MF-moment-dsdT}, which reads
	\begin{align} \label{eqn:smallRecursion}
		\frac{\partial (s'_i+s'_j)}{\partial T_{ij}} &= X_1, \qquad \frac{\partial X_1}{\partial T_{ij}} = \frac{Y_1^2}{s'_i+s'_j}, \qquad \frac{\partial Y_1}{\partial T_{ij}} = -\frac{X_1Y_1}{s'_i+s'_j}, \nonumber \\
		\frac{\partial (s'_i-s'_j)}{\partial T_{ij}} &= X_2, \qquad \frac{\partial X_2}{\partial T_{ij}} = \frac{Y_2^2}{s'_i-s'_j}, \qquad \frac{\partial Y_2}{\partial T_{ij}} = -\frac{X_2Y_2}{s'_i-s'_j}.
	\end{align}
	Since the recursive structure for $s'_i+s'_j$ and $s'_i-s'_j$ are the same, they can be written in the same form as
	\begin{equation*}
		\frac{\partial s'_{ij}}{\partial T_{ij}} = X; \qquad \frac{\partial X}{\partial T_{ij}} = \frac{Y^2}{s'_{ij}}; \qquad \frac{\partial Y}{\partial T_{ij}} = -\frac{XY}{s'_{ij}},
	\end{equation*}
	where $s'_{ij}$ denotes either $s'_i+s'_j$ or $s'_i-s'_j$.
	Finally, to prove the theorem, it is claimed that
	\begin{equation*}
		\frac{\partial^{2n} s'_{ij}}{\partial T_{ij}^{2n}} = \frac{a_0Y^{2n} + a_1Y^{2n-2}X^2 + \cdots + a_{n-1}Y^2X^{2n-2}}{{s'_{ij}}^{2n-1}},
	\end{equation*}
	where $a_m$ is an integer coefficient for $m=0,\ldots,n-1$.
	This claim is proved by induction.
	When $n=1$, it is clearly seen from \eqref{eqn:smallRecursion}.
	Suppose for case $n$ it is true, then differentiate $T_{ij}$ one more time, and after some rearrangement, it is shown that
	\begin{equation*}
		\frac{\partial^{2n+1} s'_{ij}}{\partial T_{ij}^{2n+1}} = \frac{a'_0Y^{2n}X + a'_1Y^{2n-2}X^3 + \cdots + a'_{n-1}Y^2X^{2n-1}}{\partial {s'_{ij}}^{2n}},
	\end{equation*}
	where $a'_m = (-4n+2m+1)a_m + 2(m+1)a_{m+1}$ for $m=0,\ldots,n-2$, and $a'_{n-1} = (-2n-1)a_{n-1}$.
	Then, differentiate $T_{ij}$ once again, it is shown that
	\begin{align*}
		\frac{\partial^{2n+2} s'_{ij}}{\partial T_{ij}^{2n+2}} = \frac{a''_0Y^{2n+2} + a''_1Y^{2n}X^2 + \cdots + a''_{n}Y^2X^{2n}}{{s'_{ij}}^{2n+1}},
	\end{align*}
	where $a''_0 = a'_0$, $a''_m = (-4n+2m-2)a'_{m-1} + (2m+1)a'_m$ for $m=1,\ldots,n-1$, and $a''_n = (-2n-2)a'_{n-1}$.
	This finishes the proof for the claim.
	Now, note that when evaluated at $T=0$, we have $Y=1$, $X=0$.
	Thus $\left. \partial^{2n}s'_{ij} / \partial T_{ij}^{2n} \right|_{T=0} = a_0 / s_{ij}^{2n-1}$, which finishes the proof for Theorem \ref{thm:MF-moment-specialRecursion} by noting that $a_0$ is an integer function of $n$.
\end{proof}

It should be noted that the idea in the proof of Theorem \ref{thm:MF-moment-specialRecursion} cannot be generalized to the large recursion stated in Theorem \ref{thm:MF-moment-dsdT}, due to the omitted terms in the derivatives of $X$ and $Y$.
Even in this simpler case, giving a non-recursive expression for $a(n)$ is complicated, as seen from the recursive formula given for $a_m(n)$ in the proof.

\chapter{The Second and Third Order Moment of Matrix Fisher Distribution} \label{app:MF-moment-second-third}

In this Appendix, non-recursive expressions of the second and third order moments are provided based on the development in Chapter \ref{section:MF-moments}.

\section{The Second Order Moments}

Based on Lemma \ref{lemma:MF-moment-EQii}, the second order moment of the form $\expect{Q_{ii}Q_{jj}}$ can be calculated as
\begin{align} \label{eqn:MF-moment-EQiijj}
	\expect{Q_{ii}Q_{jj}} = \frac{1}{c(S)} \frac{\partial^2 c(S)}{\partial s_i \partial s_j},
\end{align}
for any $i,j\in\{1,2,3\}$.
Next, using the recursion developed in Theorem \ref{thm:MF-moment-dsdT}, the second order moment of the forms $\expect{Q_{ij}Q_{ij}}$ and $\expect{Q_{ij}Q_{ji}}$ can be calculated as
\begin{subequations} \label{eqn:MF-moment-EQijij-EQijji}
	\begin{align}
		\expect{Q_{ij}Q_{ij}} &= \frac{1}{c(S)}\left(-\frac{\partial c(S)}{\partial s_i}\frac{s_i}{s_j^2-s_i^2}+\frac{\partial c(S)}{\partial s_j}\frac{s_j}{s_j^2-s_i^2}\right), \\
		\expect{Q_{ij}Q_{ji}} &= \frac{1}{c(S)}\left(-\frac{\partial c(S)}{\partial s_i}\frac{s_j}{s_j^2-s_i^2}+\frac{\partial c(S)}{\partial s_j}\frac{s_i}{s_j^2-s_i^2}\right),
	\end{align}
\end{subequations}
for any $i\neq j \in \{1,2,3\}$, and $|s_i|\neq |s_j|$.

If $s_i = s_j \neq 0$, \eqref{eqn:MF-moment-EQijij-EQijji} can be evaluated by taking the limit $s_j\to s_i$, and their explicit expressions are
\begin{subequations} \label{eqn:MF-moment-EQijij-EQijji-degenerate1}
	\begin{align}
		\expect{Q_{ij}Q_{ij}} &= \frac{1}{c(S)}\frac{1}{2s_i} \left( \frac{\partial c(S)}{\partial s_i} + s_i\left( \frac{\partial^2 c(S)}{\partial s_i^2} - \frac{\partial^2 c(S)}{\partial s_i \partial s_j} \right) \right), \\
		\expect{Q_{ij}Q_{ji}} &= \frac{1}{c(S)}\frac{1}{2s_i}\left( -\frac{\partial c(S)}{s_i} + s_i\left( \frac{\partial^2 c(S)}{\partial s_i^2} - \frac{\partial^2 c(S)}{\partial s_i \partial s_j} \right) \right).
	\end{align}
\end{subequations}
Similarly, if $s_i = -s_j \neq 0$, their expressions become
\begin{subequations}
	\begin{align}
		\expect{Q_{ij}Q_{ij}} &= \frac{1}{c(S)} \frac{1}{2s_i} \left( \frac{\partial c(S)}{\partial s_i}  + s_i\left( \frac{\partial^2 c(S)}{\partial s_i^2} + \frac{\partial^2 c(S)}{\partial s_i \partial s_j} \right) \right), \\
		\expect{Q_{ij}Q_{ji}} &= \frac{1}{c(S)} \frac{1}{2s_i} \left( \frac{\partial c(S)}{\partial s_i} - s_i\left( \frac{\partial^2 c(S)}{\partial s_i^2} + \frac{\partial^2 c(S)}{\partial s_i \partial s_j} \right) \right).
	\end{align}
\end{subequations}
If $s_i = s_j = 0$, the above equations can also be evaluated by taking the limit $s_i\to 0$, as
\begin{subequations} \label{eqn:MF-moment-EQijij-EQijji-degenerate3}
	\begin{align}
		\expect{Q_{ij}Q_{ij}} &= \frac{1}{c(S)} \frac{\partial^2 c(S)}{\partial s_i^2}, \\
		\expect{Q_{ij}Q_{ji}} &= -\frac{1}{c(S)} \frac{\partial^2 c(S)}{\partial s_i \partial s_j}.
	\end{align}
\end{subequations}

Given these results and Theorem \ref{thm:MF-moment-dcds}, the second order derivatives of the normalizing constant can be easily evaluated by solving a linear system using the first order derivatives.
This result is summarized in the following theorem.
\begin{theorem} \label{thm:MF-moment-dcds-second}
	Let $\partial^2 c(S) \in \mathbb{R}^{3\times 3}$ be a matrix whose elements are the second order derivatives $\partial^2 c(S)/\partial s_i \partial s_j$, $i,j\in\{1,2,3\}$.
	Then $\partial^2 c(S)$ satisfies a linear system $A \cdot \mathrm{vec}(\partial^2 c(S)) = b$, where $A\in\mathbb{R}^{9\times 9}$ is constant, and $b\in\mathbb{R}^9$ only involves $S$, $c(S)$, and the first order derivatives of $c(S)$.
\end{theorem}
\begin{proof}
	According to \eqref{eqn:MF-moment-EQiijj}, it can be shown that
	\begin{align*}
		\frac{\partial^2 c(S)}{\partial s_i^2} = c(S)\expect{Q_{ii}^2} = c(S)\left(1-\expect{Q_{ij}^2}-\expect{Q_{ik}^2}\right),
	\end{align*}
	where $i\neq j\neq k$.
	If $s_j\neq s_k$, then by \eqref{eqn:MF-moment-EQijij-EQijji} and the above equation, $\partial^2 c(S)/\partial s_i^2$ can be written as an expression involved with $S$, $c(S)$, and the first order derivatives of $c(S)$.
	If $s_j=s_k\neq 0$, or $s_j=-s_k\neq 0$, or $s_j=s_k=0$, substitute \eqref{eqn:MF-moment-EQijij-EQijji-degenerate1} to \eqref{eqn:MF-moment-EQijij-EQijji-degenerate3} into the above equation.
	Then the coefficient of $\partial^2 c(S)/\partial s_i^2$, $\partial^2 c(S)/\partial s_i \partial s_j$, and $\partial^2 c(S)/\partial s_i \partial s_k$ on the right hand side are constants.
	After moving them to the left hand side, the right hand side only has $S$, $c(S)$, and the first order derivatives of $c(S)$.
	These provides three equations in total.
	
	Next, by \eqref{eqn:MF-moment-EQiijj} and \eqref{eqn:SO3-Rkk}, it can be shown that
	\begin{align*}
		\frac{\partial^2 c(S)}{\partial s_i \partial s_j} = c(S)\expect{Q_{ii}Q_{kk}} = c(S)\left(\expect{Q_{kk}}+\expect{Q_{ij}Q_{ji}}\right),
	\end{align*}
	for $i\neq j\neq k$.
	Substitute \eqref{eqn:MF-S2D} and one from \eqref{eqn:MF-moment-EQijij-EQijji} to \eqref{eqn:MF-moment-EQijij-EQijji-degenerate3} into the above equation, and move $\partial^2 c(S)/\partial s_i^2$, $\partial^2 c(S)/\partial s_i \partial s_j$ from the right had side to the left hand side, the remaining six equations are obtained.
\end{proof}

\section{The Third Order Moments}

Based on Lemma \ref{lemma:MF-moment-EQii}, the third order moments of the form
$\expect{Q_{ii}Q_{jj}Q_{kk}}$ are
\begin{align}
	\expect{Q_{ii}Q_{jj}Q_{kk}} = \frac{1}{c(S)} \frac{\partial^3 c(S)}{\partial s_i \partial s_j \partial s_k},
\end{align}
for $i,j,k\in\{1,2,3\}$.
Next, moments of the form $\expect{Q_{ii}Q_{jk}Q_{kj}}$ are
\begin{subequations}
	\begin{align}
		\expect{Q_{ii}Q_{jk}Q_{jk}} &= \frac{1}{c(S)} \left( \frac{\partial^2 c(S)}{\partial s_i \partial s_k} \frac{s_k}{s_k^2-s_j^2} - \frac{\partial^2 c(S)}{\partial s_i \partial s_j} \frac{s_j}{s_k^2-s_j^2} \right), \\
		\expect{Q_{ii}Q_{jk}Q_{kj}} &= \frac{1}{c(S)} \left( \frac{\partial^2 c(S)}{\partial s_i \partial s_k} \frac{s_j}{s_k^2-s_j^2} - \frac{\partial^2 c(S)}{\partial s_i \partial s_j} \frac{s_k}{s_k^2-s_j^2} \right),
	\end{align}
\end{subequations}
for $i\neq j\neq k$ and $|s_j|\neq |s_k|$.
Moments of the form $\expect{Q_{jj}Q_{jk}Q_{jk}}$ are
\begin{subequations}
	\begin{align}
		\expect{Q_{jj}Q_{jk}Q_{jk}} &= \frac{1}{c(S)} \left( \frac{\partial^2 c(S)}{\partial s_j \partial s_k} \frac{s_k}{s_k^2-s_j^2} - \frac{\partial^2 c(S)}{\partial s_j^2} \frac{s_j}{s_k^2-s_j^2} \right. \nonumber \\
		&\qquad\qquad \left. - \frac{\partial c(S)}{\partial s_j} \frac{s_j^2+s_k^2}{(s_j^2-s_k^2)^2} + \frac{\partial c(S)}{\partial s_k} \frac{2s_js_k}{(s_j^2-s_k^2)^2} \right), \\
		\expect{Q_{jj}Q_{jk}Q_{kj}} &= \frac{1}{c(S)} \left( \frac{\partial^2 c(S)}{\partial s_j \partial s_k} \frac{s_j}{s_k^2-s_j^2} - \frac{\partial^2 c(S)}{\partial s_j^2} \frac{s_k}{s_k^2-s_j^2} \right. \nonumber \\
		&\qquad\qquad \left. - \frac{\partial c(S)}{\partial s_j} \frac{2s_js_k}{(s_j^2-s_k^2)^2} + \frac{\partial c(S)}{\partial s_k} \frac{s_j^2+s_k^2}{(s_j^2-s_k^2)^2} \right),
	\end{align}
\end{subequations}
for $j\neq k$ and $|s_j|\neq |s_k|$.
Finally, moments of the form $\expect{Q_{ij}Q_{jk}Q_{ki}}$ are
\begin{subequations}
	\begin{align}
		\expect{Q_{ij}Q_{jk}Q_{ki}} &= \frac{1}{c(S)}\left( \frac{\partial c(S)}{\partial s_i} \frac{s_js_k}{(s_i^2-s_j^2)(s_i^2-s_k^2)} \right. \nonumber \\
		&\qquad\qquad \left. + \frac{\partial c(S)}{\partial s_j} \frac{s_is_k}{(s_j^2-s_i^2)(s_j^2-s_k^2)} + \frac{\partial c(S)}{\partial s_k} \frac{s_is_j}{(s_k^2-s_i^2)(s_k^2-s_j^2)} \right), \\
		\expect{Q_{ij}Q_{jk}Q_{ik}} &= \frac{1}{c(S)}\left( \frac{\partial c(S)}{\partial s_i} \frac{s_is_j}{(s_j^2-s_i^2)(s_k^2-s_i^2)} \right. \nonumber \\
		&\qquad\qquad \left. + \frac{\partial c(S)}{\partial s_j} \frac{s_j^2}{(s_i^2-s_j^2)(s_k^2-s_j^2)} + \frac{\partial c(S)}{\partial s_k} \frac{s_js_k}{(s_j^2-s_k^2)(s_i^2-s_k^2)} \right),
	\end{align}
\end{subequations}
for $i\neq j\neq k$ and $|s_i|\neq |s_j| \neq |s_k|$.

\chapter{Non-uniqueness Parameters of MFG} \label{app:MFG-unique}

In this appendix, it is shown that how the MFG can be parameterized differently if $S$ has repeated values.
First, Definition \ref{def:psvd} is augmented by the following uniqueness condition:
\textit{the first nonzero element of each column of $U'$ is positive} \cite{khatri1977mises}.
This condition ensures that the columns of $U$ and $V$ cannot undergo simultaneous sign changes.
Then the equivalence of MFG parameterizations is given in the following theorem.
\begin{theorem} \label{thm:MFG-equivalent}
	Suppose $F=USV$, $\tilde{F}=\tilde{U}\tilde{S}\tilde{V}^T$ are the proper SVD of $F$ and $\tilde{F}$ with the augmented uniqueness condition.
	Then  $\mathcal{MG}(\mu,\allowbreak \Sigma,\allowbreak P,\allowbreak U,\allowbreak S,\allowbreak V)$ and $\mathcal{MG}(\tilde{\mu},\allowbreak \tilde{\Sigma},\allowbreak \tilde{P},\allowbreak \tilde{U},\allowbreak \tilde{S},\allowbreak \tilde{V})$ are equivalent if and only if $\mu=\tilde{\mu}$, $S=\tilde{S}$, and one of the following conditions is satisfied:
	\begin{enumerate}
		\item if $s_1=s_2=s_3=0$, then $\Sigma = \tilde{\Sigma}$. \label{case:s1=s2=s3=0}
		\item if $s_1 \neq s_2=s_3=0$, then \label{case:s1 neq s2=s3=0}
		\begin{enumerate}
			\item [2I.] $\exists \theta_1, \theta_2 \in \mathbb{R}$ such that $\tilde{U}=UT_1$, $\tilde{V}=VT_2$ where $T_1 = \exp(\theta_1\hat{e}_1)$ and $T_2 = \exp(\theta_2\hat{e}_1)$, $[\tilde{P}_{:,2},\tilde{P}_{:,3}] = [P_{:,2},P_{:,3}]\begin{bmatrix} \cos\theta_1 & -\sin\theta_1 \\ \sin\theta_1 & \cos\theta_1\end{bmatrix}$ where $P_{:,i}$ is the $i$-th column of $P$, and $\Sigma = \tilde{\Sigma}$.
			\item [2B.] $\exists \theta_1, \theta_2 \in \mathbb{R}$ such that $\tilde{U}=UT_1$, $\tilde{V}=VT_2$ where $T_1 = \exp(\theta_1\hat{e}_1)$ and $T_2 = \exp(\theta_2\hat{e}_1)$, $[\tilde{P}_{:,2},\tilde{P}_{:,3}] = [P_{:,2},P_{:,3}]\begin{bmatrix} \cos\theta_2 & -\sin\theta_2 \\ \sin\theta_2 & \cos\theta_2\end{bmatrix}$ where $P_{:,i}$ is the $i$-th column of $P$, and $\Sigma = \tilde{\Sigma}$.
		\end{enumerate} 
		\item if $s_1=s_2=s_3 \neq 0$, then $\exists T\in\SO{3}$ such that $\tilde{U}=UT$, $\tilde{V}=VT$, $\tilde{P}=PT$ and $\Sigma=\tilde{\Sigma}$.\label{case:s1=s2=s3 neq 0}
		\item if $s_1 \neq s_2=s_3 \neq 0$, then $\exists \theta\in\mathbb{R}$ such that $\tilde{U}=UT$, $\tilde{V}=VT$, $\tilde{P}=PT$, where $T=\exp(\theta\hat{e}_1)$, and $\Sigma=\tilde{\Sigma}$.\label{case:s1 neq s2=s3 neq 0}
		\item if $s_1=s_2 \neq |s_3|$, then $\exists \theta\in\mathbb{R}$ such that $\tilde{U}=UT$, $\tilde{V}=VT$, $\tilde{P}=PT$, where $T=\exp(\theta\hat{e}_3)$, and $\Sigma=\tilde{\Sigma}$.\label{case:s1=s2 neq |s3|}
		\item if $s_1 \neq s_2 \neq |s_3|$, then $U=\tilde{U}$, $V=\tilde{V}$, $P=\tilde{P}$, and $\Sigma=\tilde{\Sigma}$. \label{case:s1 neq s2 neq |s3|}
		\item if $s_1=s_2=-s_3 \neq 0$, let $L = \diag(1,1,-1)$, then \label{case:s1=s2=-s3 neq 0}
		\begin{enumerate}
			\item [7I.] $\exists T\in\SO{3}$ such that $\tilde{U}=UT$, $\tilde{V}=VLTL$, $\tilde{P}=PT$, and \linebreak $\tilde{\Sigma} = \Sigma + P\big[T(\tr{S}I_{3\times 3}-S)T^T - (\tr{S}I_{3\times 3}-S)\big]P^T$.
			\item [7B.] $\exists T\in\SO{3}$ such that  $\tilde{U}=ULTL$, $\tilde{V}=VT$, $\tilde{P}=PT$, and \linebreak $\tilde{\Sigma} = \Sigma + P\big[ T(\tr{S}I_{3\times 3}-S)T^T - (\tr{S}I_{3\times 3}-S) \big]P^T$.
		\end{enumerate}
		\item if $s_1 \neq s_2=-s_3 \neq 0$, then \label{case:s1 neq s2=-s3 neq 0}
		\begin{enumerate}
			\item [8I.] $\exists \theta\in\mathbb{R}$ such that $\tilde{U}=UT$, $\tilde{V}=VT^T$, $\tilde{P}=PT$, and $\tilde{\Sigma} = \Sigma + P\big[ T(\tr{S}I_{3\times 3}-S)T^T - (\tr{S}I_{3\times 3}-S) \big]P^T$, where $T=\exp(\theta\hat{e}_1)$.
			\item [8B.] $\exists \theta\in\mathbb{R}$ such that $\tilde{U}=UT^T$, $\tilde{V}=VT$, $\tilde{P}=PT$, and $\tilde{\Sigma} = \Sigma + P\big[ T(\tr{S}I_{3\times 3}-S)T^T - (\tr{S}I_{3\times 3}-S) \big]P^T$, where $T=\exp(\theta\hat{e}_1)$.
		\end{enumerate}
	\end{enumerate}
	For cases \ref{case:s1 neq s2=s3=0}), \ref{case:s1=s2=s3 neq 0}) and \ref{case:s1 neq s2=-s3 neq 0}), ``$I$'' means the condition holds for MFGI, and ``$B$'' means the condition holds for MFGB.
\end{theorem}
\begin{proof}
	By Lemma \ref{lemma:MFG-equivalent-intermediate}, the two MFGs are equivalent if and only if $F=\tilde{F}$, $\mu_c=\tilde{\mu}_c$ for all $R\in\SO{3}$, and $\Sigma=\tilde{\Sigma}_c$.
	
	The conditions provided in the theorem begin with $S=\tilde {S}$. 
	We first consider additional conditions to make them equivalent to $F=\tilde{F}$.
	Since a matrix cannot have two sets of different singular values, $F=\tilde{F}$ implies $S=\tilde{S}$.
	For case \ref{case:s1=s2=s3=0}), $S=\tilde{S}$ trivially implies $F=\tilde {F}$ as they are all zeros.
	For case \ref{case:s1 neq s2 neq |s3|}), $F=\tilde{F}$ if and only if their proper SVDs are the same, due to the augmented uniqueness condition for SVD when the singular values are different.
	In short, $F=\tilde{F}$ if and only if $S=\tilde{S}$ for cases 1) and 6).
	However, when $F$ has repeated singular values, $U$ and $V$ are only unique up to a rotation.
	We consider three possibilities regarding the multiplicity of $S$.
	
	(i) If $s_1 \neq s_2=s_3=0$, corresponding to case \ref{case:s1 neq s2=s3=0}) in the theorem, then $F=\tilde{F}$ if and only if $U\diag(s_1,0,0)V^T = \tilde{U}\diag(s_1,0,0)\tilde{V}^T$.
	This means the first column of $U$ and $V$ is unique, while other columns can be arbitrarily chosen as long as $U,V\in\SO{3}$, which is the same as the conditions on $U,V$ in case \ref{case:s1 neq s2=s3=0}).
	
	(ii) If $s_1=s_2$ and(or) $s_2=s_3 \neq 0$, which corresponds to case \ref{case:s1=s2=s3 neq 0}) to \ref{case:s1=s2 neq |s3|}) in the theorem, then the left and right singular vectors (columns of $U$ and $V$) w.r.t the repeated singular value form subspaces of $\mathbb{R}^3$.
	And $F=\tilde{F}$ if and only if the left and right singular vectors w.r.t the repeated singular value are rotated by the same rotation matrix, while the singular vectors w.r.t the non-repeated singular value are the same.
	The above condition on $U,V$ is the same as those given in case \ref{case:s1=s2=s3 neq 0}) to \ref{case:s1=s2 neq |s3|}).
	
	(iii) If $s_2=-s_3 \neq 0$, corresponding to cases \ref{case:s1=s2=-s3 neq 0}) and \ref{case:s1 neq s2=-s3 neq 0}) in the theorem.
	Let $S' = SL \triangleq \diag(s'_1,s'_2,s'_3)$, then $USV^T = US'LV^T = ULS'V^T$.
	Since $s'_2=s'_3$, for all $\theta\in\mathbb{R}$ and $T=\exp(\theta\hat{e}_1)$, it can be proved for MFGI that
	\begin{align*}
		US'LV^T = UTS'T^TLV^T = UTS'LLT^TLV^T = UTS(VLTL)^T = UTS(VT^T)^T,
	\end{align*}
	and for MFGB that
	\begin{align*}
		ULS'V^T = ULTS'T^TV^T = ULTLLS'T^TV^T = (ULTL)S(VT)^T = UT^TS(VT)^T,
	\end{align*}
	This shows the sufficiency and necessity for the conditions on $U,V$ in case \ref{case:s1 neq s2=-s3 neq 0}).
	For case \ref{case:s1=s2=-s3 neq 0}), $T$ can be arbitrarily chosen from $\SO{3}$, but in general $LTL \neq T^T$.
	
	Next, we consider the equivalent conditions for $\mu_c=\tilde{\mu}_c$ and $\Sigma_c=\tilde{\Sigma}_c$.
	For case \ref{case:s1=s2=s3=0}), the matrix Fisher density parts reduce to one, $\mu_c = \mu$, and $\Sigma_c=\Sigma$, so the remaining Gaussian density parts are equivalent if and only if their means and covariance matrices are the same, as shown in Lemma \ref{lemma:MFG-equivalent-intermediate}.
	Next, the three possibilities regarding the multiplicity of $S$ are considered individually as follows.
	
	For (i), let $Q=U^TRV$, since $\tilde{U}=UT_1$, $\tilde{V}=VT_2$ and $S=\tilde{S}$, it can be proved for MFGI that
	\begin{align} \label{eqn:MFG-equivalent-Miuc-MFGI}
		\tilde{\mu}_c &= \tilde{\mu} + \tilde{P}(T_1^TU^TRVT_2S-ST_2^TV^TR^TUT_1)^\vee \nonumber \\
		&= \tilde{\mu} + \tilde{P}T_1^T(U^TRVT_2ST_1^T-T_1ST_2^TV^TR^TU)^\vee \nonumber \\
		&= \tilde{\mu} + \tilde{P}T_1^T(QS-SQ^T)^\vee \nonumber \\
		&= \tilde{\mu} + \tilde{P}T_1^T[0, -s_1Q_{31}, s_1Q_{21}]^T.
	\end{align}
	And similarly, for MFGB, the above equation becomes
	\begin{align} \label{eqn:MFG-equivalent-Miuc-MFGB}
		\tilde{\mu}_c &= \tilde{\mu} + \tilde{P}(ST_1^TU^TRVT_2-T_2^TV^TR^TUT_1S)^\vee \nonumber \\
		&= \tilde{\mu} + \tilde{P}T_2^T(T_2ST_1^TU^TRV-V^TR^TUT_1ST_2^T)^\vee \nonumber \\
		&= \tilde{\mu} + \tilde{P}T_2^T(SQ-Q^TS)^\vee \nonumber \\
		&= \tilde{\mu} + \tilde{P}T_2^T[0, s_1Q_{13}, -s_1Q_{12}]^T.
	\end{align}
	Note that $Q$ and $R$ have a one-to-one correspondence, so $\mu_c = \tilde{\mu}_c$ for all $R\in\SO{3}$ is equivalent to $\mu_c = \tilde{\mu}_c$ for all $Q\in\SO{3}$.
	It is clear the conditions on $\mu$ and $P$ in case \ref{case:s1 neq s2=s3=0}) are sufficient for this.
	On the other hand, if $\mu_c = \tilde{\mu}_c$ for all $Q\in\SO{3}$, substituting $Q=I_{3 \times 3}$ into \eqref{eqn:MFG-equivalent-Miuc-MFGI} and \eqref{eqn:MFG-equivalent-Miuc-MFGB} proves $\mu=\tilde{\mu}$; substituting $Q = \expb{\tfrac{\pi}{2}e_2}$ and $Q = \expb{\tfrac{\pi}{2}e_3}$ into \eqref{eqn:MFG-equivalent-Miuc-MFGI} and \eqref{eqn:MFG-equivalent-Miuc-MFGB} proves $[\tilde{P}_{:,2},\tilde{P}_{:,3}] = [P_{:,2},P_{:,3}]\begin{bmatrix} \cos\theta_1 & -\sin\theta_1 \\ \sin\theta_1 & \cos\theta_1\end{bmatrix}$.
	Finally, denote $T'$ as either $\begin{bmatrix} \cos\theta_1 & -\sin\theta_1 \\ \sin\theta_1 & \cos\theta_1\end{bmatrix}$ or $\begin{bmatrix} \cos\theta_2 & -\sin\theta_2 \\ \sin\theta_2 & \cos\theta_2\end{bmatrix}$ then
	\begin{align}
		\tilde{P}(\tr{S}I_{3 \times 3}-S)(\tilde{P})^T = s_1 \begin{bmatrix} P_{:,2} & P_{:,3} \end{bmatrix}  T'(T')^T \begin{bmatrix} P_{:,2}^T \\  P_{:,3}^T \end{bmatrix} = P(\tr{S}I_{3 \times 3}-S)P^T.
	\end{align}
	so $\Sigma_c = \tilde{\Sigma}_c$ if and only if $\Sigma = \tilde{\Sigma}$.
	
	For (ii), a similar calculation as in \eqref{eqn:MFG-equivalent-Miuc-MFGI} and \eqref{eqn:MFG-equivalent-Miuc-MFGB} shows $\tilde{\mu}_c = \tilde{\mu} + \tilde{P}T^T \nu_R$.
	The sufficiency of the conditions on $\mu$ and $P$ in case \ref{case:s1=s2=s3 neq 0}) to \ref{case:s1=s2 neq |s3|}) follows immediately.
	For the necessary direction, substituting $Q = I_{3 \times 3}$ into the above equation proves $\mu = \tilde{\mu}$;
	substituting $Q = \expb{\tfrac{\pi}{2}\hat{e}_1}$, $Q = \expb{\tfrac{\pi}{2}\hat{e}_2}$ and $Q = \expb{\tfrac{\pi}{2}\hat{e}_3}$ proves $\tilde{P} = PT$ since $s_i+s_j>0$ for any $i \neq j$ in case \ref{case:s1=s2=s3 neq 0}) to \ref{case:s1=s2 neq |s3|}).
	Finally, note that the repeated diagonal entries of $S$ and $\tr{S}I_{3 \times 3}-S$ share the same indices, thus $T(\tr{S}I_{3 \times 3}-S)T^T = \tr{S}I_{3 \times 3}-S$.
	This proves $\Sigma_c = \tilde{\Sigma}_c$ if and only if $\Sigma = \tilde{\Sigma}$.
	The same argument in this paragraph also proves case \ref{case:s1 neq s2 neq |s3|}) in the theorem by letting $T = I_{3 \times 3}$.
	
	For (iii), a similar calculation as in \eqref{eqn:MFG-equivalent-Miuc-MFGI} and \eqref{eqn:MFG-equivalent-Miuc-MFGB} shows $\tilde{\mu}_c = \tilde{\mu} + \tilde{P}T^T \nu_R$.
	The sufficiency of the conditions on $\mu$ and $P$ in case \ref{case:s1=s2=-s3 neq 0}) and \ref{case:s1 neq s2=-s3 neq 0}) is clear.
	For necessity, substituting $Q=I_{3\times 3}$ proves $\mu=\tilde{\mu}$; substituting $Q=\begin{bmatrix} -1 & 0 & 0 \\ 0 & 0 & 1 \\ 0 & 1 & 0 \end{bmatrix}$, $Q=\begin{bmatrix} 0 & 0 & 1 \\ 0 & -1 & 0 \\ 1 & 0 & 0 \end{bmatrix}$, and $Q=\expb{\tfrac{\pi}{2}\hat{e}_3}$ proves $\tilde{P} = PT$, because $s_2-s_3>0$, $s_1-s_3>0$, and $s_1+s_2>0$ in these two cases.
	Finally, $\tilde{\Sigma}_c$ is
	\begin{equation}
		\tilde{\Sigma}_c = \tilde{\Sigma} + PT(\tr{S}I-S)T^TP^T.
	\end{equation}
	However, different from (ii), $T(\tr{S}I_{3 \times 3}-S)T^T \neq \tr{S}I_{3 \times 3}-S$.
	And as a result, $\Sigma_c = \tilde{\Sigma}_c$ if and only if $\tilde{\Sigma} = \Sigma + P\left[ T(\tr{S}I-S)T^T - (\tr{S}I-S) \right]P^T$.
\end{proof}

In other words, Theorem \ref{thm:MFG-equivalent} states that if $F$ has repeated singular values and $s_3\geq 0$, then MFG can be parameterized differently by rotating $U$, $V$, and $P$ in a consistent way.
This is intuitively straightforward by looking at how the correlation $P$ is constructed in Theorem \ref{thm:MFG-construction}.
When $s_3<0$, it is noticeable that $\Sigma$ and $\tilde{\Sigma}$ are also different as indicated in case \ref{case:s1=s2=-s3 neq 0}) and \ref{case:s1 neq s2=-s3 neq 0}).
The reason for this is when $s_2=-s_3$, $T(\tr{S}I_{3\times 3}-S)T^T \neq \tr{S}I_{3\times 3}-S$.

